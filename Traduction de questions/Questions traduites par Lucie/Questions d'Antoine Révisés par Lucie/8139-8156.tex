\documentclass[letterpaper, 12pt]{article}
\usepackage[francais]{babel}

\usepackage{amsmath,amsfonts,amsthm,amssymb, graphicx,wasysym,multirow}
\usepackage[latin1]{inputenc}

\pagestyle{plain}

\setlength{\topmargin}{-2cm}
\setlength{\textheight}{23.5cm}
\setlength{\textwidth}{18cm}
\setlength{\oddsidemargin}{-1cm}
\setlength{\parindent}{0pt}

%\pdfoutput=1

\begin{document}

8139-- How do we call a period of time of ten years?\\
\\
a) Year\\
b) Decade\\
c) Millenium\\
d) Century\\

Answer : b$)$\\

Feedback :\\
A \textbf{decade} is a period of ten years. \\
The answer is b$)$.\\
\\
\\
8140-- How do we call a period of time of 100 years?\\
\\
a) Year\\
b) Century\\
c) Millenium\\
d) Decade\\

Answer : b$)$\\

Feedback :\\
A \textbf{century} is a period of 100 years. \\
The answer is b$)$.\\
\\
\\
8141-- How do we call a period of time of 1000 years?\\
\\
a) Year\\
b) Decade\\
c) Millenium\\
d) Century\\

Answer : c$)$\\

Feedback :\\
A \textbf{millenium} is a period of time equal to 1000 years. \\
The answer is c$)$.\\
\\
\\
8142-- How do we call a period of time equal to 12 000 months?\\
\\
a) a year\\
b) a decade\\
c) a millenium\\
d) a century\\

Answer : c$)$\\

Feedback :\\
There are 12 months in a year. So, you need do divide 12 000 by 12 to find the equivalent number of years, which is 1 000 years. A \textbf{millenium} is a period of time equal to 1000 years. \\
The answer is c$)$.\\
\\
\\
8143-- How do we call a period of time equal to 36 525 days?\\
\\
a) a year\\
b) a decade\\
c) a millenium\\
d) a century\\

Answer : d$)$\\

Feedback :\\
There are 365 days in a year, except one year every four years, that has a day more. Therefore, there are 36 500 days in 100 regular years, and 100 consecutive years include 25 leap years. So, 36 500 days + 25 days = 36 525 days = 100 years. A \textbf{century} is a period of time equal to 100 years. \\
The answer is d$)$.\\
\\
\\
8144-- Which of these period of time corresponds to approximately 520 weeks?\\
\\
a) a year\\
b) a decade\\
c) a millenium\\
d) a century\\

Answer : b$)$\\

Feedback :\\
There are approximately 52 weeks in a year. Therefore, there are approximately 520 weeks in 10 years. A \textbf{decade} is a period of time equal to 10 years. \\
The answer is b$)$.\\

8145-- How may days are there \textbf{between} March 25th and April 10th?\\
\\
a) 13 days\\
b) 15 days\\
c) 17 days\\
d) 19 days\\

Answer : b$)$\\

Feedback :\\
There are 15 days between March 25th and April 10th: from March 26th to 31st and from April $1^{\texttt{st}}$ to 9th. \\
The answer is b$)$.\\
\\
\\
8146-- How many days are there \textbf{between} a Thursday and the next Tuesday?\\
\\
a) 4 days\\
b) 5 days\\
c) 6 days\\
d) 7 days\\

Answer : a$)$\\

Feedback :\\
There are 4 days between a Thursday and the next Tuesday: Friday, Saturday, Sunday and Monday. \\
The answer is a$)$.\\
\\
\\
8147-- How many months are there \textbf{between} September and the next month of May?\\
\\
a) 4 months\\
b) 5 months\\
c) 6 months\\
d) 7 months\\

Answer : d$)$\\

Feedback :\\
There are 7 months between the month of September and the next month of May: October, November, December, January, February, March, and April. \\
The answer is d$)$.\\
\\
\\
8148-- How many years were there \textbf{between} 1986 and 2005?\\
\\
a) 12 years\\
b) 16 years\\
c) 18 years\\
d) 22 years\\

Answer : c$)$\\

Feedback :\\
You have to start counting from 1986, until they end of the year 2004. So, 18 years have passed. \\
The answer is c$)$.\\
\\
\\
8149-- How many days are there \textbf{between} the months of June and September?\\
\\
a) 61 years\\
b) 62 years\\
c) 91 years\\
d) 92 years\\

Answer : b$)$\\

Feedback :\\
You have to start counting from the first day of July, Until the end of August. It is a period of 2 months. Both these months last 31 days. $31+31=62$ days \\
The answer is b$)$.\\
\\
\\
8150-- How many days were there \textbf{between} 2007 and 2009?\\
\\
a) 365 days\\
b) 366 days\\
c) 730 days\\
d) 1095 days\\

Answer : b$)$\\

Feedback :\\
The issue here is to know the number of days during 2008. Years that are multiples of 4 are leap years. So, in 2008 there were 366 years. \\
The answer is b$)$.\\
\\
\\
8150-- How many days were there \textbf{between} 2007 and 2009?\\
\\
a) 365 days\\
b) 366 days\\
c) 730 days\\
d) 1095 days\\

Answer : b$)$\\

Feedback :\\
The issue at hand is to know the number of days in the year 2008. Years that are multiples of 4 are leap years. The year 2008 had 366 days. \\
The answer is b$)$.\\

8151-- My neighbor's dog is 12 years old, while mine is 5 years old. What is the age difference between our dogs?\\
\\
a) 5 years\\
b) 7 years\\
c) 9 years\\
d) 12 years\\

Answer : b$)$\\

Feedback :\\
To calculate the difference, we need to subtract the lowest number from the greatest number. \\
$12-5=7$ years\\
The answer is b$)$.\\
\\
\\
8152-- How many days are there \textbf{between} Halloween and Christmas?\\
\\
a) 54 days\\
b) 61 days\\
c) 87 days\\
d) 128 days\\

Answer : a$)$\\

Feedback :\\
Halloween is always on October 31st, and Christmas is always on December 25th. Thus, we need to calculate the sum of the 30 days of November with the 24 first days of December to find the answer.\\
$30+24=54$ days\\
The answer is a$)$.\\
\\
\\
8153-- How many days are there \textbf{between} Christmas and Valentine's Day?\\
\\
a) 40 days\\
b) 50 days\\
c) 60 days\\
d) 70 days\\

Answer : b$)$\\

Feedback :\\
Christmas is on December 25th, and Valentine's day is always on February 14th. Thus, we have to calculate the sum of the 31 days of January with the 6 last days of December and the 13 first days of February to get the answer.\\
$6+31+13=50$ days\\
The answer is b$)$.\\
\\
\\
8154-- The winner of a cycling competition got to the finish line after 34 minutes. The second cyclist got there after 37 minutes and ten seconds. The third one got to the finish line 15 seconds after the second one. How much time ahead was the winner, from the second cyclist?\\
\\
a) 30 seconds\\
b) 3 minutes\\
c) 3 minutes and 10 seconds\\
d) 3 minutes and 25 seconds\\

Answer : c$)$\\

Feedback :\\
To find the defference between the two, you need to subtract the time of the first one from the time of the second one.\\
37 minutes and 10 seconds$-34$ minutes = 3 minutes et 10 secondes.\\
The answer is c$)$.\\
\\
\\
8155-- On a particular spring day, the sun rises at 6:38 am and sets at 19:05 pm. What is the period of sunshine during that day?\\
\\
a) 12:27\\
b) 13:17\\
c) 13:42\\
d) 13:47\\

Answer : a$)$\\

Feedback :\\
The sun rises at 6:38 am.\\
22 minutes later, it is 7:00 am;\\
12 hours later, it is 19:00 pm;\\
5 minutes later, the sun stes at 19:05 pm.\\
12 hours + 22 minutes + 5 minutes = 12:27\\
The answer is a$)$.\\
\\
\\
8156-- Peter was born on October 4th. How many days are there \textbf{between} his birthday and Christmas?\\
\\
a) 35 days\\
b) 46 days\\
c) 56 days\\
d) 81 days\\

Answer : d$)$\\

Feedback :\\
After his birthday, there are 27 days left in Obtober (from 5th to 31st),\\
30 days in November (the whole month),\\
and the 24 first days of December.\\
27 days + 30 days + 24 days = 81 days\\
The answer is d$)$.\\

\end{document} 