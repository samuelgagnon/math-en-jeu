\documentclass[letterpaper, 12pt]{article}
\usepackage[french]{babel}

\usepackage{amsmath,amsfonts,amsthm,amssymb,graphicx,multirow,hyperref,color}
\usepackage[latin1]{inputenc}

\pagestyle{plain}

\setlength{\topmargin}{-2cm}
\setlength{\textheight}{23.5cm}
\setlength{\textwidth}{12cm}
\setlength{\oddsidemargin}{-1cm}
\setlength{\parindent}{0pt}

\begin{document}


7300-- Which of the following is an even number?\\

a) 125 648\\
b) 385 333\\
c) 568 369\\
d) 777 777\\

Answer : a)\\

Feedback :\\
The number 125 648 is evenly divisable by 2, because the 
units digit is divisable by 2 without remainder.\\
\begin{eqnarray*}
\frac{8}{2}=4\\
\end{eqnarray*}
Since 125 648 is evenly divisable by 2, it is an even number.\\
The answer is a).\\



7301-- Which of the following is an odd number?\\

a) 5648\\
b) 85 334\\
c) 568 362\\
d) 777 777\\

Answer : d)\\

Feedback :\\
The number 777 777 is not evenly divisable by 2, because the units 
digit is not divisable by 2 without remainder.\\
\begin{eqnarray*}
\frac{7}{2}=3,5\\
\end{eqnarray*}
Since 777 777 is not evenly divisable by 2, it is an odd number.\\
The answer is d).\\




7302-- Which of the following is an even number?\\

a) 1641\\
b) 5333\\
c) 854 532\\
d) 999 997\\

Answer : c)\\

Feedback :\\
The number 854 532 is evenly divisable by 2, because the units digit is divisable by 2 without remainder.\\
\begin{eqnarray*}
\frac{2}{2}=1\\
\end{eqnarray*}
Since 854 532 is evenly divisable by 2, it is an even number.\\
The answer is c).\\




7303-- Which of the following is an odd number?\\

a) 128 566\\
b) 333 334\\
c) 659 999\\
d) 777 772\\

Answer : c)\\

Feedback :\\
The number 659 999 is not evenly divisable by 2, because the 
units digit is not divisable by 2 without remainder.\\
\begin{eqnarray*}
\frac{9}{2}=4,5\\
\end{eqnarray*}
Since 659 999 is not evenly divisable by 2, it is an odd number.\\
The answer is c).\\

7304-- Complete the following sentence.\\
\begin{center}
Any number is divisable by four without remainder if the number 
created by the two last digits is evenly divisable by  \underline{\quad\quad}.\\
\end{center}

a) Ten\\
b) Eleven\\
c) Four\\
d) One\\

Answer : c)\\

Feedback:\\
\begin{center}
Any number is divisable by four without remainder if the number created by the two last digits is evenly divisable by \textbf{four}.\\
\end{center}

Exemple: 7316
\begin{eqnarray*}
\frac{16}{4}=4\\
\end{eqnarray*}
\begin{eqnarray*}
\frac{7316}{4}=1829\\
\end{eqnarray*}
The answer is c).\\



7305-- In the number 123.45, which digit is number 5?\\

a) hundredths digit\\
b) tenths digit\\
c) hundreds digit\\
d) units digit\\

Answer : a)\\

Feedback :\\
\begin{center}
% use packages: array
\begin{tabular}{|rrr|rrr|rrr|}
\hline
\multicolumn{6}{|c|}{integers} &\multicolumn{3}{|c|}{decimals} \\
\hline
\multicolumn{3}{|c|}{Class of} &\multicolumn{3}{|c|}{Class of} &  \multicolumn{3}{c|}{} \\
\multicolumn{3}{|c|}{thousands} &\multicolumn{3}{|c|}{units} &  \multicolumn{3}{c|}{} \\
\hline
H & T & U &H & T & U, & T\up{th} & \textbf{H\up{th}} & K\up{th} \\
\hline
\hline
&  &  & 1 & 2 & 3, & 4 & \textbf{5} &  \\
\hline
\end{tabular}
\end{center}

\scriptsize
\begin{center}
% use packages: array
\begin{tabular}{ll}
H : Hundreds & T\up{th} : Tenths\\
T : Tens & H\up{th} :Hundredths\\
U : Units & K\up{e} : Thousandths\\
\end{tabular}
\end{center}

\normalsize

The answer is a).\\




7306-- Express the following multiplication in terms of powers.\\ 
\begin{center}
$5\times5\times5$\\.
\end{center}

a) $ 2^{1}$\\
b) $ 3^{3}$\\
c) $ 5^{3}$\\
d) $ 5^{4}$\\

Answer : c)\\

Feedback :\\
$ 5^{3}$ = $5\times5\times5$\\
The answer is c).\\




7307-- Express the following multiplication in terms of powers.\\ 
\begin{center}
$7\times4\times7\times4$\\
\end{center}

a) $ 4^{4}$\\
b) $ 7^{4}$\\
c) $ 4^{2}\times7^{2}$\\
d) $ 10^{5}\times10^{8}$\\

Answer : c)\\

Feedback :\\
First of all, because of the commutativity of multiplications, the same numbers are put together. 
Afterwards, only one number from each group is kept and the powers are added.

\begin{eqnarray*}
7\times4\times7\times4
&=&4\times4\times7\times7\\
&=&4^{2}\times7^{2}\\                                                                           \end{eqnarray*}
The answer is c).\\




7308-- Which number is represented by the addition?\\ 
\begin{center}
800 000 + 70 000 + 500 + 20 + 3\\
\end{center}

a) 32 578\\
b) 78 523\\
c) 87 523\\
d) 870 523\\

Answer : d)\\

Feedback :\\
\begin{center}
% use packages: array
\begin{tabular}{|rrr|rrr|rrr|}
\hline
\multicolumn{6}{|c|}{integers} &\multicolumn{3}{|c|}{decimals} \\
\hline
\multicolumn{3}{|c|}{Class of} &\multicolumn{3}{|c|}{Class of} &  \multicolumn{3}{c|}{} \\
\multicolumn{3}{|c|}{thousands} &\multicolumn{3}{|c|}{units} &  \multicolumn{3}{c|}{} \\
\hline
H & T & U &H & T & U, & T\up{th} & \textbf{H\up{th}} & K\up{th} \\
\hline
\hline
8 & 0 & 0 & 0 & 0 & 0 &  & &\\
 & 7 & 0 & 0 & 0 & 0 &  & &\\
+ &  &  & 5 & 0 & 0 &  & &\\
 &  &  &  & 2 & 0 &  & &\\
 &  &  &  &  & 3 &  & &\\
\hline
\hline
 8 & 7 & 0 & 5 & 2 & 3 &  & &
\\
\hline
\end{tabular}
\end{center}

\scriptsize
\begin{center}
% use packages: array
\begin{tabular}{ll}
H : Hundreds & T\up{th} : Tenths\\
T : Tens & H\up{th} :Hundredths\\
U : Units & K\up{e} : Thousandths\\
\end{tabular}
\end{center}

\normalsize
The answer is d).\\




7309-- Sort in ascending order.\\ 
\begin{center}
101, 8997, 10 722\\
\end{center}

a) 101, 8997, 10 722\\
b) 101, 10 722, 8997\\
c) 8997, 10 722, 101\\
d) 10 722, 101, 8997\\\\

Answer : a)\\

Feedback :\\
\begin{center}
% use packages: array
\begin{tabular}{|rrr|rrr|rrr|}
\hline
\multicolumn{6}{|c|}{integers} &\multicolumn{3}{|c|}{decimals} \\
\hline
\multicolumn{3}{|c|}{Class of} &\multicolumn{3}{|c|}{Class of} &  \multicolumn{3}{c|}{} \\
\multicolumn{3}{|c|}{thousands} &\multicolumn{3}{|c|}{units} &  \multicolumn{3}{c|}{} \\
\hline
H & T & U &H & T & U, & T\up{th} & \textbf{H\up{th}} & K\up{th} \\
\hline
\hline
 &  &  & 1 & 0 & 1 &  &  & \\
 &  & 8 & 9 & 9 & 7 &  &  &\\
 & 1 & 0 & 7 & 2 & 2 &  &  &\\
\hline
\end{tabular}
\end{center}

\scriptsize
\begin{center}
% use packages: array
\begin{tabular}{ll}
H : Hundreds & T\up{th} : Tenths\\
T : Tens & H\up{th} :Hundredths\\
U : Units & K\up{e} : Thousandths\\
\end{tabular}
\end{center}

\normalsize

The answer is a).\\

7310-- Continue the following sequence of numbers.\\
\begin{center} 
50 000, 100 000, 150 000, 200 000, 250 000, \underline{\quad\quad}\\
\end{center}

a) 170 000\\
b) 300 000\\
c) 350 000\\
d) 1 000 000\\

Answer : b)\\

Feedback :\\
The consistency is of + 50 000.\\

\begin{center}
$50\ 000+50\ 000=100\ 000$\\
$100\ 000+50\ 000=150\ 000$\\
$150\ 000+50\ 000=200\ 000$\\
$200\ 000+50\ 000=250\ 000$\\
$250\ 000+50\ 000=\textbf{300\ 000}$\\\end{center}
The answer is b).\\



7311-- Norah is making Christmas cards. Each box contains 50 cards. 
If Norah made 5 boxes of cards, how any cards did she make?\\

a) 25\\
b) 250\\
c) 500\\
d) 1000\\

Answer : b)\\

Feedback :\\
If there are 50 cards in 1 box, there are 250 cards in 5 boxes.\\

\begin{center}
% use packages: array
\begin{tabular}{cccc}
 & & 5 & 0\\
x & & & 5\\
\cline {2-4}
& 2 & 5 & 0\\
\end{tabular}
\end{center}
The answer is b).\\




7312-- Find the mistake in the multiplication table.\\

\begin{center}
% use packages: array
\begin{tabular}{c|cc}
X & 1 & 10 \\
\hline
2 & 2 & 20\\
4 & 4 & 30\\
6 & 6 & 60\\

\end{tabular}
\end{center}

a) 2\\
b) 6\\
c) 30\\
d) 60\\

Answer : c)\\

Feedback :\\
\begin{center}
% use packages: array
\begin{tabular}{c|cc}
X & 1 & \textbf{10} \\
\hline
2 & 2 & 20\\
\textbf{4} & 4 & \textbf{40}\\
6 & 6 & 60\\

\end{tabular}
\end{center}


\begin{center}
% use packages: array
\begin{tabular}{ccc}
 & 1 & 0\\
X & & 4\\
\cline {2-3}
 & 4 & 0\\
\end{tabular}
\end{center}
The answer is c).\\




7313-- What is the result of the addition 5101 + 315?\\

a) 5315\\
b) 5325\\
c) 5416\\
d) 5425\\

Answer : c)\\

Feedback :\\
\begin{center}
% use packages: array
\begin{tabular}{ccccc}
& 5 & 1 & 0 & 1\\
+ &  & 3 & 1 & 5\\
\cline {2-5}
& 5 & 4 & 1 & 6\\
\end{tabular}
\end{center}
The answer is c).\\




7314-- Ludovic has five dollars. Marco has three times the amount of dollars Ludovic has. How much money does Marco have? Identify the operation you would use to solve this problem.\\

a) addition\\
b) division\\
c) multiplication\\
d) subtraction\\

Answer : c)\\

Feedback :\\
Marco has three times \textbf{more} money than Ludovic.\\
\begin{center}
$3\times 5 = 15$
\end{center}
The answer is c).\\




7315-- Marianne's dog had 6 puppies. Three puppies have 
already been adopted. How many puppies does she still 
have to give away? Identify the operation you would use 
to solve this problem.\\

a) addition\\
b) division\\
c) multiplication\\
d) subtraction\\

Answer : d)\\

Feedback :\\
Three puppies have been adopted on a total of six puppies. So, three puppies have been removed out of the six. 
Therefore there are three puppies left to give away.\\
\begin{center}
$6-3 = 3$
\end{center}
The answer is d).\\




7316-- Jeremy paid 60\$ on video games that were sold 20\$ each. How many games did he buy? 
Identify the operation you would use to solve this problem. \\

a) addition\\
b) division\\
c) multiplication\\
d) subtraction\\

Answer : b)\\

Feedback :\\
By dividing the total amount spent by the price of one game, we get the number of games.\\

\begin{eqnarray*}
60�20=3\\
\end{eqnarray*}
The answer is b).\\




7317-- Charli is ten years old. Mia is six years older than him. 
Identify the operation you would use to find Mia's age. \\

a) addition\\
b) division\\
c) multiplication\\
d) subtraction\\

Answer : a)\\

Feedback :\\
Mia is six years \textbf{older} than Charli.
\begin{center}
$10 + 6 = 16$
\end{center}
The answer is a).\\




7318-- How much money will Harry have to spend to buy 3 sweaters at 25\$ each? Keep in mind that you need to solve this problem with \textbf{one operation} only. Identify the operation you need to solve the problem. \\

a) addition\\
b) division\\
c) multiplication\\
d) subtraction\\

Answer: c)\\

Feedback :\\
For 3 sweaters at 25\$ each, Harry will have to spend 3 \textbf{times} the amount of 25\$, therefore 75\$.\\
\begin{center}
$3\times25 = 75$
\end{center}
The answer is c).\\


7319-- For Christmas, Maya got six gifts. Isa\"i got four gifts more than her. How many gifts did Isa\"i get?.
Identify the operation you need to solve this problem. \\

a) addition\\
b) division\\
c) multiplication\\
d) subtraction\\

Answer : a)\\

Feedback :\\
Isa\"i got four gifts \textbf{more} than Maya.
\begin{center}
$6+4 = 10$
\end{center}
The answer is a).\\


7320-- Jules has to travel 178 km by bike to get to his cabin. At noon he had travelled 63 km. How many kilometers 
does he still have to travel? Identify the operation you would need to solve this problem. \\

a) addition\\
b) division\\
c) multiplication\\
d) subtraction\\

Answer : d)\\

Feedback :\\
Jules has to travel a total of 178 km and he has already travelled 63. So he has \textbf{less} kilometers left to do than when he started.

\begin{center}
$178-63 = 115$
\end{center}
The answer is d).\\




7321-- The bicycle path in Boisjoli is 112 km long. There is a rest area every 14 kilometers. How many rest areas are there along this path? Identify the operation you would need to solve the problem. \\

a) addition\\
b) division\\
c) multiplication\\
d) subtraction\\

Answer : b)\\

Feedback :\\
By dividing the total number of kilometers by the number of kilometers that separate each rest area, we get the total number of rest areas.\\

\begin{eqnarray*}
\frac{112}{14}=8\\
\end{eqnarray*}
The answer is b).\\




7322-- Choose the most realistic number to complete the sentence.\\
\begin{center}
\'Emile got \underline{\quad\quad}\% on his writing exam.\\
\end{center}

a) -5\\
b) -0,25\\
c) 2$\frac{2}{3}$\\
d) 85\\

Answer : d)\\

Feedback :\\
\begin{center}
\'Emile got \textbf{85}\% on his writing exam.\\
\end{center}

Unless you have a very peculiar teacher, the results of an exam can't be a negative number nor a percentage over 100\%. In fact, a percentage (\%) is only a fraction of 100. We use it so often that mathematicians gave it a name.\\
The answer is d).\\

7323-- Choose the most realistic number to complete the sentence.\\ 
\begin{center}
Julie put \underline{\quad\quad} cups of brown sugar in her cookie recipe. \\
\end{center}

a) -1\\
b) -0,25\\
c) 2$\frac{1}{4}$\\
d) 500\\

Answer : c)\\

Feedback :\\
\begin{center}
Julie put \textbf{2$\frac{1}{4}$} cups of brown sugar in her cookie recipe.\\
\end{center}

First of all, negative numbers are eliminated. Next, the number 500 is way too big to be realistic.\\
The answer is c).\\



7324-- Choose the most realistic number to complete the sentence.\\ 
\begin{center}
Mary's little sister is celebrating her\underline{\quad\quad} $^{th}$ anniversary.\\
\end{center}

a) -1\\
b) -0,25\\
c) 2$\frac{1}{4}$\\
d) 11\\

Answer : d)\\

Feedback :\\
\begin{center}
Mary's little sister is celebrating her \textbf{11$^{th}$} anniversary.\\
\end{center}

First of all, negative numbers are not possible. Next, the fraction is to be eliminated because we don't celebrate
the fraction of a year.\\
The answer is d).\\


7325-- Choose the most realistic number to complete the sentence.\\
\begin{center}
A bubblegum costs \underline{\quad\quad}\$.\\
\end{center}

a) -5\\
b) 0,25\\
c) 20\\
d) 100\\

Answer : b)\\

Feedback :\\
\begin{center}
A bubblegum costs \textbf{0,25}\$.\\
\end{center}

First, prices are always positive numbers, so -5 is impossible. Also, the numbers 20 and 100 are way too high to be realistic.\\
The answer is b).\\


7326-- Place the number 983 502 in the correct set.

\begin{center}
    \includegraphics[width=6cm]{Q7326.eps}
% Q7326.eps : 300dpi, width=3.39cm, height=3.39cm, bb=0 0 400 400
    \end{center}


a) Set A\\
b) Set B\\
c) Set C\\
d) Set D\\

Answer : c)\\

Feedback :\\
\begin{center}
% use packages: array
\begin{tabular}{|rrr|rrr|rrr|}
\hline
\multicolumn{6}{|c|}{integers} &\multicolumn{3}{|c|}{decimals} \\
\hline
\multicolumn{3}{|c|}{Class of} &\multicolumn{3}{|c|}{Class of} &  \multicolumn{3}{c|}{} \\
\multicolumn{3}{|c|}{thousands} &\multicolumn{3}{|c|}{units} &  \multicolumn{3}{c|}{} \\
\hline
H & T & U &H & T & U, & T\up{th} & \textbf{H\up{th}} & K\up{th} \\
\hline
\hline
&  & 1 & 0 & 0 & 0 & 0 & 0 & 0 &  &  &  \\
 &  &  & 9 & 8 & 3 & 5 & 0 & 2  &  &  & \\
 &  &  &  &  &  &  &  & 0  &  &  & \\
\hline
\end{tabular}
\end{center}

\tiny
\begin{center}
% use packages: array
\begin{tabular}{ll}
H : Hundreds & T\up{th} : Tenths\\
T : Tens & H\up{th} :Hundredths\\
U : Units & K\up{e} : Thousandths\\
\end{tabular}
\end{center}

\normalsize

To begin with, since the number 983 502 is smaller than \mbox{1 000 000}, it is part of set A.\\

Next, the number 983 502 does not have a 5 in the hundreds of thousands position and does not have a 2 in the units position.\\

The number 983 502 is then part of set C.\\
The answer is c).\\


7327-- Which number has the highest amount of tens of thousands? \\

a) 193 000\\
b) 560 000\\
c) 851 888\\
d) 910 124\\

Answer : d)\\

Feedback :\\

\begin{center}
% use packages: array
\begin{tabular}{|rrr|rrr|rrr|}
\hline
\multicolumn{6}{|c|}{integers} &\multicolumn{3}{|c|}{decimals} \\
\hline
\multicolumn{3}{|c|}{Class of} &\multicolumn{3}{|c|}{Class of} &  \multicolumn{3}{c|}{} \\
\multicolumn{3}{|c|}{thousands} &\multicolumn{3}{|c|}{units} &  \multicolumn{3}{c|}{} \\
\hline
H & T & U &H & T & U, & T\up{th} & \textbf{H\up{th}} & K\up{th} \\
\hline
\hline
 \textbf{1} & \textbf{9} & 3 & 0 & 0 & 0 & & & \\
 \textbf{5} & \textbf{6} & 0 & 0 & 0 & 0 & & &\\
 \textbf{8} & \textbf{5} & 1 & 8 & 8 & 8 & & &\\
 \textbf{9} & \textbf{1} & 0 & 1 & 2 & 4 & & &\\
\hline
\end{tabular}
\end{center}

\tiny
\begin{center}
% use packages: array
\begin{tabular}{ll}
H : Hundreds & T\up{th} : Tenths\\
T : Tens & H\up{th} :Hundredths\\
U : Units & K\up{e} : Thousandths\\
\end{tabular}
\end{center}

\normalsize

The number 910 124 has 91 tens of thousands.\\
The answer is d).\\



7328--Which number represents \textit{nine hundred twenty nine thousand and three hundred}? \\

a) 129 300\\
b) 222 222\\
c) 300 929\\
d) 929 300\\

Answer : d)\\

Feedback :\\
\begin{center}
% use packages: array
\begin{tabular}{|rrr|rrr|rrr|}
\hline
\multicolumn{6}{|c|}{integers} &\multicolumn{3}{|c|}{decimals} \\
\hline
\multicolumn{3}{|c|}{Class of} &\multicolumn{3}{|c|}{Class of} &  \multicolumn{3}{c|}{} \\
\multicolumn{3}{|c|}{thousands} &\multicolumn{3}{|c|}{units} &  \multicolumn{3}{c|}{} \\
\hline
H & T & U &H & T & U, & T\up{th} & \textbf{H\up{th}} & K\up{th} \\
\hline
\hline
\multicolumn{3}{|c|}{neuf cent} &  \multicolumn{3}{|c|}{} & & &\\
\multicolumn{3}{|c|}{vingt-neuf} &  \multicolumn{3}{|c|}{trois cents} & & &\\
\multicolumn{3}{|c|}{mille} &  \multicolumn{3}{|c|}{} & & &\\
 9 & 2 & 9 & 3 & 0 & 0 & & &\\
\hline
\end{tabular}
\end{center}

\tiny
\begin{center}
% use packages: array
\begin{tabular}{ll}
H : Hundreds & T\up{th} : Tenths\\
T : Tens & H\up{th} :Hundredths\\
U : Units & K\up{e} : Thousandths\\
\end{tabular}
\end{center}

\normalsize

The answer is d).\\




7329--Round each number to the nearest unit and do the operation.\\

\begin{eqnarray*}
9,64 \times 2,01\\
\end{eqnarray*}

a) 2\\
b) 9\\
c) 12\\
d) 20\\

Answer: d)

Feedback :\\
First, we round up the number 9,64 to 10, because the number in the tenths position is greater or equal to 5.\\

Next, we round down the number 2,01 to 2, because the number in the tenths position is smaller than 5.\\

Finally, we do the multiplication.\\

\begin{center}
 \begin{tabular}{r r}
 & 10\\
$\times$ & 2\\
\hline
 & 20
\end{tabular}
\end{center}
The answer is d).\\

7330-- Which portion of the figure is colored?\\

\begin{center}
\includegraphics[width=5cm]{Q7330.eps}
% Q7330.eps : 300dpi, width=3.39cm, height=3.39cm, bb=0 0 400 400
\end{center}

a) $\frac{1}{4}$\\

b) $\frac{1}{2}$\\

c) $\frac{2}{3}$\\

d) 1\\

Answer : b)\\

Feedback :\\
First, the figure is a rectangle separated into two right triangles by its diagonal. Next, there is one triangle out of two that is colored. Therefore, the colored portion is $\frac{1}{2}$.\\
The answer is b).\\


7331--What is the sum of 3587,29 + 0,10?\\

a) 3587,39\\
b) 3588,29\\
c) 3597,39\\
d) 4587,29\\

Answer : a)\\

Feedback :\\
\begin{center}
% use packages: array
\begin{tabular}{|rrr|rrr|rrr|}
\hline
\multicolumn{6}{|c|}{integers} &\multicolumn{3}{|c|}{decimals} \\
\hline
\multicolumn{3}{|c|}{Class of} &\multicolumn{3}{|c|}{Class of} &  \multicolumn{3}{c|}{} \\
\multicolumn{3}{|c|}{thousands} &\multicolumn{3}{|c|}{units} &  \multicolumn{3}{c|}{} \\
\hline
H & T & U &H & T & U, & T\up{th} & \textbf{H\up{th}} & K\up{th} \\
\hline
\hline
&  & 3 & 5 & 8 & 7, & 2 & 9 &  \\
 & + &  &  &  & 0, & 1 & 0 &  \\
\hline
\hline
 &  & 3 & 5 & 8 & 7, & 3 & 9 &  \\
\hline
\end{tabular}
\end{center}

\scriptsize
\begin{center}
% use packages: array
\begin{tabular}{ll}
H : Hundreds & T\up{th} : Tenths\\
T : Tens & H\up{th} :Hundredths\\
U : Units & K\up{e} : Thousandths\\
\end{tabular}
\end{center}

\normalsize

The answer is a).\\


7332-- Today, the outside temperature is of $-5^{o}$C. If we predict a drop of $5^{o}$C for tomorrow, what will the temperature \mbox{be}\\

a) $-10^{o}$C\\
b) $-5^{o}$C\\
c) $0^{o}$C\\
d) $5^{o}$C\\

Answer : a)\\

Feedback :\\
\begin{center}
\includegraphics[width=5cm]{Q7332.eps}
% Q7332.eps : 300dpi, width=3.39cm, height=3.39cm, bb=0 0 400 400
\end{center}

We withdraw 5 to the number $- 5$.\\
\begin{eqnarray*}
-5 - 5 = - 10\\
\end{eqnarray*}

The answer is a).\\





7333-- Which fraction of the set is represented by the suns?

\begin{center}
    \includegraphics[width=6cm]{Q7333.eps}
% Q7333.eps : 300dpi, width=3.39cm, height=3.39cm, bb=0 0 400 400
    \end{center}


a) $\frac{1}{3}$\\\\
b) $\frac{1}{2}$\\\\
c) $\frac{4}{6}$\\\\
d) $\frac{6}{3}$\\\\

Answer : b)\\

Feedback :\\
On a set of six items, there are three suns.\\
\begin{eqnarray*}
\frac{3}{6}\\
\end{eqnarray*}

If we reduce $\frac{3}{6}$ to its simplest form, we get $\frac{1}{2}$.\\

\begin{center}
\begin{tabular}{c}

$\div 3$  \\
\LARGE $\curvearrowright$  \\
\Large $\frac{3}{6}$  =  \Large $\frac{1}{2}$ \\
\rotatebox{180}{\LARGE$\curvearrowleft$}\\
$\div 3$  \\

\end{tabular}
\end{center}
The answer is  b).\\




7334-- Considering the order of operation, by which operation do we have to start?\\
\begin{eqnarray*}
20+10\times3\div1-4\\
\end{eqnarray*}


a) $1-4$\\
b) $3\div1$\\
c) $10\times3$\\
d) $20+10$\\

Answer : c)\\

Feedback :\\
First, to determine the order of operations, we need to look from left to right. Then, we have to start by the first multiplication or division we come accross.\\
\begin{eqnarray*}
10\times3\\
\end{eqnarray*}
The answer is c).\\



7335-- Considering the order of operations, by which operation should we start?\\

\begin{eqnarray*}
20+10\times3\div(1-4)\\
\end{eqnarray*}

a) $1-4$\\
b) $3\div1$\\
c) $10\times3$\\
d) $20+10$\\

Answer : a)\\

Feedback :\\
We have to start by the operation in parentheses.\\
\begin{eqnarray*}
1-4.
\end{eqnarray*}
The answer is a).\\

7336-- Which number is represented by the addition?\\
\begin{center}
100 000 + 20 000 + 1000 + 200 + 10 + 2
\end{center}

a) 12 221\\
b) 121 212\\
c) 300 000\\
d) 429 200\\

Answer : b)\\

Feedback :\\
\begin{center}
% use packages: array
\begin{tabular}{|rrr|rrr|rrr|}
\hline
\multicolumn{6}{|c|}{integers} &\multicolumn{3}{|c|}{decimals} \\
\hline
\multicolumn{3}{|c|}{Class of} &\multicolumn{3}{|c|}{Class of} &  \multicolumn{3}{c|}{} \\
\multicolumn{3}{|c|}{thousands} &\multicolumn{3}{|c|}{units} &  \multicolumn{3}{c|}{} \\
\hline
H & T & U &H & T & U, & T\up{th} & \textbf{H\up{th}} & K\up{th} \\
\hline
\hline
1 & 0 & 0 & 0 & 0 & 0 &  &  &  \\
  & 2 & 0 & 0 & 0 & 0 &  &  &  \\
+ &   & 1 & 0 & 0 & 0 &  &  &  \\
  &   &   & 2 & 0 & 0 &  &  &  \\
  &   &   &   & 1 & 0 &  &  &  \\
  &   &   &   &   & 2 &  &  &  \\
\hline
\hline
1 & 2 & 1 & 2 & 1 & 2 &  &  &  \\
\hline
\end{tabular}
\end{center}

\scriptsize
\begin{center}
% use packages: array
\begin{tabular}{ll}
H : Hundreds & T\up{th} : Tenths\\
T : Tens & H\up{th} :Hundredths\\
U : Units & K\up{e} : Thousandths\\
\end{tabular}
\end{center}

\normalsize
The answer is b).\\



7337-- Which number is represented by the addition?\\
\begin{center}
70 000 + 1 000 + 400 + 30
\end{center}

a) 7143\\
b) 71 430\\
c) 71 433\\
d) 704 300\\

Answer : b)\\

Feedback :\\
\begin{center}
% use packages: array
\begin{tabular}{|rrr|rrr|rrr|}
\hline
\multicolumn{6}{|c|}{integers} &\multicolumn{3}{|c|}{decimals} \\
\hline
\multicolumn{3}{|c|}{Class of} &\multicolumn{3}{|c|}{Class of} &  \multicolumn{3}{c|}{} \\
\multicolumn{3}{|c|}{thousands} &\multicolumn{3}{|c|}{units} &  \multicolumn{3}{c|}{} \\
\hline
H & T & U &H & T & U, & T\up{th} & \textbf{H\up{th}} & K\up{th} \\
\hline
\hline
  & 7 & 0 & 0 & 0 & 0 &  &  & \\
  &   & 1 & 0 & 0 & 0 &  &  & \\
+ &   &   & 4 & 0 & 0 &  &  & \\
  &   &   &   & 3 & 0 &  &  & \\
\hline
\hline
  & 7 & 1 & 4 & 3 & 0 &  &  & \\
\hline
\end{tabular}
\end{center}

\scriptsize
\begin{center}
% use packages: array
\begin{tabular}{ll}
H : Hundreds & T\up{th} : Tenths\\
T : Tens & H\up{th} :Hundredths\\
U : Units & K\up{e} : Thousandths\\
\end{tabular}
\end{center}

\normalsize

The answer is b).\\


7338-- Which number is represented by the addition?\\
\begin{center}
600 000 + 60 000 + 6000 + 60 + 6
\end{center}

a) 66 666\\
b) 660 666\\
c) 666 066\\
d) 666 666\\

Answer : c)\\

Feedback :\\
\begin{center}
% use packages: array
\begin{tabular}{|rrr|rrr|rrr|}
\hline
\multicolumn{6}{|c|}{integers} &\multicolumn{3}{|c|}{decimals} \\
\hline
\multicolumn{3}{|c|}{Class of} &\multicolumn{3}{|c|}{Class of} &  \multicolumn{3}{c|}{} \\
\multicolumn{3}{|c|}{thousands} &\multicolumn{3}{|c|}{units} &  \multicolumn{3}{c|}{} \\
\hline
H & T & U &H & T & U, & T\up{th} & \textbf{H\up{th}} & K\up{th} \\
\hline
\hline
6 & 0 & 0 & 0 & 0 & 0 & & &\\
  & 6 & 0 & 0 & 0 & 0 & & &\\
+ &   & 6 & 0 & 0 & 0 & & &\\
  &   &   &   & 6 & 0 & & &\\
  &  &    &   &   & 6 & & &\\
\hline
\hline
6 & 6 & 6 & 0 & 6 & 6 & & &\\
\hline
\end{tabular}
\end{center}

\scriptsize
\begin{center}
% use packages: array
\begin{tabular}{ll}
H : Hundreds & T\up{th} : Tenths\\
T : Tens & H\up{th} :Hundredths\\
U : Units & K\up{e} : Thousandths\\
\end{tabular}
\end{center}

\normalsize

The answer is c).\\


7339-- Which number is represented by the addition?\\
\begin{center}
5 000 + 400 + 10 + 3
\end{center}

a) 314\\
b) 513\\
c) 5000\\
d) 5413\\

Answer : d)\\

Feedback :\\
\begin{center}
% use packages: array
\begin{tabular}{|rrr|rrr|rrr|}
\hline
\multicolumn{6}{|c|}{integers} &\multicolumn{3}{|c|}{decimals} \\
\hline
\multicolumn{3}{|c|}{Class of} &\multicolumn{3}{|c|}{Class of} &  \multicolumn{3}{c|}{} \\
\multicolumn{3}{|c|}{thousands} &\multicolumn{3}{|c|}{units} &  \multicolumn{3}{c|}{} \\
\hline
H & T & U &H & T & U, & T\up{th} & \textbf{H\up{th}} & K\up{th} \\
\hline
\hline
 &   & 5 & 0 & 0 & 0 & & &\\
 &   &   & 4 & 0 & 0 & & &\\
 & + &   &   & 1 & 0 & & &\\
 &   &   &   &   & 3 & & &\\
\hline
\hline
 &   & 5 & 4 & 1 & 3 & & &\\
\hline
\end{tabular}
\end{center}

\scriptsize
\begin{center}
% use packages: array
\begin{tabular}{ll}
H : Hundreds & T\up{th} : Tenths\\
T : Tens & H\up{th} :Hundredths\\
U : Units & K\up{e} : Thousandths\\
\end{tabular}
\end{center}

\normalsize
The answer is d).\\

7340-- Which number is represented by the addition?\\
\begin{center}
100 000 + 1000 + 100 + 10
\end{center}

a) 111\\
b) 11 110\\
c) 101 110\\
d) 111 111\\

Answer : c)\\

Feedback :\\
\begin{center}
% use packages: array
\begin{tabular}{|rrr|rrr|rrr|}
\hline
\multicolumn{6}{|c|}{integers} &\multicolumn{3}{|c|}{decimals} \\
\hline
\multicolumn{3}{|c|}{Class of} &\multicolumn{3}{|c|}{Class of} &  \multicolumn{3}{c|}{} \\
\multicolumn{3}{|c|}{thousands} &\multicolumn{3}{|c|}{units} &  \multicolumn{3}{c|}{} \\
\hline
H & T & U &H & T & U, & T\up{th} & \textbf{H\up{th}} & K\up{th} \\
\hline
\hline
1 & 0 & 0 & 0 & 0 & 0 & & & \\
  &   & 1 & 0 & 0 & 0 & & & \\
+ &   &   & 1 & 0 & 0 & & & \\
  &   &   &   & 1 & 0 & & & \\
\hline
\hline
1 & 0 & 1 & 1 & 1 & 0 & & & \\
\hline
\end{tabular}
\end{center}

\scriptsize
\begin{center}
% use packages: array
\begin{tabular}{ll}
H : Hundreds & T\up{th} : Tenths\\
T : Tens & H\up{th} :Hundredths\\
U : Units & K\up{e} : Thousandths\\
\end{tabular}
\end{center}

\normalsize
The answer is c).\\

7341-- Which number is represented by the addition?\\
\begin{center}
400 000 + 50 000 + 1000 + 800
\end{center}

a) 4518\\
b) 8154\\
c) 451 800\\
d) 800 154\\

Answer : c)\\

Feedback :\\
\begin{center}
% use packages: array
\begin{tabular}{|rrr|rrr|rrr|}
\hline
\multicolumn{6}{|c|}{integers} &\multicolumn{3}{|c|}{decimals} \\
\hline
\multicolumn{3}{|c|}{Class of} &\multicolumn{3}{|c|}{Class of} &  \multicolumn{3}{c|}{} \\
\multicolumn{3}{|c|}{thousands} &\multicolumn{3}{|c|}{units} &  \multicolumn{3}{c|}{} \\
\hline
H & T & U &H & T & U, & T\up{th} & \textbf{H\up{th}} & K\up{th} \\
\hline
\hline
4 & 0 & 0 & 0 & 0 & 0 & & &\\
  & 5 & 0 & 0 & 0 & 0 & & &\\
+ &   & 1 & 0 & 0 & 0 & & &\\
  &   &   & 8 & 0 & 0 & & &\\
\hline
\hline
4 & 5 & 1 & 8 & 0 & 0 & & &\\
\hline
\end{tabular}
\end{center}

\scriptsize
\begin{center}
% use packages: array
\begin{tabular}{ll}
H : Hundreds & T\up{th} : Tenths\\
T : Tens & H\up{th} :Hundredths\\
U : Units & K\up{e} : Thousandths\\
\end{tabular}
\end{center}

\normalsize
The answer is c).\\


7342-- Which number is represented by the addition?\\
\begin{center}
30 000 + 600 + 20
\end{center}

a) 263\\
b) 362\\
c) 30 026\\
d) 30 620\\

Answer : d)\\

Feedback :\\
\begin{center}
% use packages: array
\begin{tabular}{|rrr|rrr|rrr|}
\hline
\multicolumn{6}{|c|}{integers} &\multicolumn{3}{|c|}{decimals} \\
\hline
\multicolumn{3}{|c|}{Class of} &\multicolumn{3}{|c|}{Class of} &  \multicolumn{3}{c|}{} \\
\multicolumn{3}{|c|}{thousands} &\multicolumn{3}{|c|}{units} &  \multicolumn{3}{c|}{} \\
\hline
H & T & U &H & T & U, & T\up{th} & \textbf{H\up{th}} & K\up{th} \\
\hline
\hline
 & 3 & 0 & 0 & 0 & 0 & & & \\
 & + &   & 6 & 0 & 0 & & & \\
 &   &   &   & 2 & 0 & & & \\
\hline
\hline
 & 3 & 0 & 6 & 2 & 0 & & & \\
\hline
\end{tabular}
\end{center}

\scriptsize
\begin{center}
% use packages: array
\begin{tabular}{ll}
H : Hundreds & T\up{th} : Tenths\\
T : Tens & H\up{th} :Hundredths\\
U : Units & K\up{e} : Thousandths\\
\end{tabular}
\end{center}

\normalsize
The answer is d).\\


7343-- Which number is represented by the addition?\\
\begin{center}
5000 + 800 + 30 + 5
\end{center}

a) 5385\\
b) 5835\\
c) 8535\\
d) 583 500\\

Answer : b)\\

Feedback :\\
\begin{center}
% use packages: array
\begin{tabular}{|rrr|rrr|rrr|}
\hline
\multicolumn{6}{|c|}{integers} &\multicolumn{3}{|c|}{decimals} \\
\hline
\multicolumn{3}{|c|}{Class of} &\multicolumn{3}{|c|}{Class of} &  \multicolumn{3}{c|}{} \\
\multicolumn{3}{|c|}{thousands} &\multicolumn{3}{|c|}{units} &  \multicolumn{3}{c|}{} \\
\hline
H & T & U &H & T & U, & T\up{th} & \textbf{H\up{th}} & K\up{th} \\
\hline
\hline
 &   & 5 & 0 & 0 & 0 & & &\\
 &   &   & 8 & 0 & 0 & & &\\
 & + &   &   & 3 & 0 & & &\\
 &   &   &   &   & 5 & & &\\
\hline
\hline
 &   & 5 & 8 & 3 & 5 & & &\\
\hline
\end{tabular}
\end{center}

\scriptsize
\begin{center}
% use packages: array
\begin{tabular}{ll}
H : Hundreds & T\up{th} : Tenths\\
T : Tens & H\up{th} :Hundredths\\
U : Units & K\up{e} : Thousandths\\
\end{tabular}
\end{center}

\normalsize
The answer is b).\\

7344-- Which number is represented by th addition?\\
\begin{center}
300 000 + 80 000 + 5000 + 900 + 90 + 7
\end{center}

a) 9597\\
b) 88 097\\
c) 385 997\\
d) 799 583\\

Answer : c)\\

Feedback :\\
\begin{center}
% use packages: array
\begin{tabular}{|rrr|rrr|rrr|}
\hline
\multicolumn{6}{|c|}{integers} &\multicolumn{3}{|c|}{decimals} \\
\hline
\multicolumn{3}{|c|}{Class of} &\multicolumn{3}{|c|}{Class of} &  \multicolumn{3}{c|}{} \\
\multicolumn{3}{|c|}{thousands} &\multicolumn{3}{|c|}{units} &  \multicolumn{3}{c|}{} \\
\hline
H & T & U &H & T & U, & T\up{th} & \textbf{H\up{th}} & K\up{th} \\
\hline
\hline
3 & 0 & 0 & 0 & 0 & 0 & & &\\
  & 8 & 0 & 0 & 0 & 0 & & &\\
  &   & 5 & 0 & 0 & 0 & & &\\
  &   &   & 9 & 0 & 0 & & &\\
  &   &   &   & 9 & 0 & & &\\
  &   &   &   &   & 7 & & &\\
\hline
\hline
3 & 8 & 5 & 9 & 9 & 7 & & &\\
\hline
\end{tabular}
\end{center}

\scriptsize
\begin{center}
% use packages: array
\begin{tabular}{ll}
H : Hundreds & T\up{th} : Tenths\\
T : Tens & H\up{th} :Hundredths\\
U : Units & K\up{e} : Thousandths\\
\end{tabular}
\end{center}

\normalsize
The answer is c).\\

7345-- Which one is an even number?\\

a) 3985\\
b) 4914\\
c) 14 815\\
d) 420 423\\

Answer : b)\\

Feedback :\\
The number 4914 is evenly divisable by  2, because the units digit is divisable by 2 without remainder.\\
\begin{eqnarray*}
\frac{4}{2}=2\\
\end{eqnarray*}
Since 4914 is evenly divisable by 2, it is an even number.\\
The answer is b).\\


7346-- Which one is an odd number?\\

a) 4903\\
b) 4904\\
c) 4916\\
d) 4918\\

Answer : a)\\

Feedback :\\
The number 4903 is not evenly divisable by 2, because the units digit is not divisable by 2 without remainder.\\
\begin{eqnarray*}
\frac{3}{2}=1,5\\
\end{eqnarray*}
Since 4903 is not evenly divisable by 2, it is an odd number.\\
The answer is a).\\

7347-- What is the position of the digit 6 in the number 212,26?\\

a) hundreds\\
b) hundredths\\
c) tenths\\
d) units\\

Answer : b)\\

Feedback :\\
\begin{center}
% use packages: array
\begin{tabular}{|rrr|rrr|rrr|}
\hline
\multicolumn{6}{|c|}{integers} &\multicolumn{3}{|c|}{decimals} \\
\hline
\multicolumn{3}{|c|}{Class of} &\multicolumn{3}{|c|}{Class of} &  \multicolumn{3}{c|}{} \\
\multicolumn{3}{|c|}{thousands} &\multicolumn{3}{|c|}{units} &  \multicolumn{3}{c|}{} \\
\hline
H & T & U &H & T & U, & T\up{th} & \textbf{H\up{th}} & K\up{th} \\
\hline
\hline
& & & 2 & 1 & 2, & 2 & \textbf{6} &  \\
\hline
\end{tabular}
\end{center}

\scriptsize
\begin{center}
% use packages: array
\begin{tabular}{ll}
H : Hundreds & T\up{th} : Tenths\\
T : Tens & H\up{th} :Hundredths\\
U : Units & K\up{e} : Thousandths\\
\end{tabular}
\end{center}

\normalsize
The answer is b).\\


7348-- What is the position of digit 4 in the number 859,46?\\

a) hundreds\\
b) hundredths\\
c) tenths\\
d) tens\\

Answer : c)\\

Feedback :\\
\begin{center}
% use packages: array
\begin{tabular}{|rrr|rrr|rrr|}
\hline
\multicolumn{6}{|c|}{integers} &\multicolumn{3}{|c|}{decimals} \\
\hline
\multicolumn{3}{|c|}{Class of} &\multicolumn{3}{|c|}{Class of} &  \multicolumn{3}{c|}{} \\
\multicolumn{3}{|c|}{thousands} &\multicolumn{3}{|c|}{units} &  \multicolumn{3}{c|}{} \\
\hline
H & T & U &H & T & U, & T\up{th} & \textbf{H\up{th}} & K\up{th} \\
\hline
\hline
 & & & 8 & 5 & 9, & \textbf{4} & 6 & \\
\hline
\end{tabular}
\end{center}

\scriptsize
\begin{center}
% use packages: array
\begin{tabular}{ll}
H : Hundreds & T\up{th} : Tenths\\
T : Tens & H\up{th} :Hundredths\\
U : Units & K\up{e} : Thousandths\\
\end{tabular}
\end{center}

\normalsize
The answer is c).\\

7349-- What is the position of the digit 1 in the number 756,19?\\

a) Hundreds\\
b) hundredths\\
c) tenths\\
d) units\\

Answer : c)\\

Feedback :\\
\begin{center}
% use packages: array
\begin{tabular}{|rrr|rrr|rrr|}
\hline
\multicolumn{6}{|c|}{integers} &\multicolumn{3}{|c|}{decimals} \\
\hline
\multicolumn{3}{|c|}{Class of} &\multicolumn{3}{|c|}{Class of} &  \multicolumn{3}{c|}{} \\
\multicolumn{3}{|c|}{thousands} &\multicolumn{3}{|c|}{units} &  \multicolumn{3}{c|}{} \\
\hline
H & T & U &H & T & U, & T\up{th} & \textbf{H\up{th}} & K\up{th} \\
\hline
\hline
 & & & 7 & 5 & 6, & \textbf{1} & 9 & \\
\hline
\end{tabular}
\end{center}

\scriptsize
\begin{center}
% use packages: array
\begin{tabular}{ll}
H : Hundreds & T\up{th} : Tenths\\
T : Tens & H\up{th} :Hundredths\\
U : Units & K\up{e} : Thousandths\\
\end{tabular}
\end{center}

\normalsize
The answer is c).\\


7350-- Which digit is 9 in the number 319,56?\\

a) hundreds digit\\
b) hundredths digit\\
c) tenths digit\\
d) units digit\\

Answer : d)\\

Feedback :\\
\begin{center}
% use packages: array
\begin{tabular}{|rrr|rrr|rrr|}
\hline
\multicolumn{6}{|c|}{integers} &\multicolumn{3}{|c|}{decimals} \\
\hline
\multicolumn{3}{|c|}{Class of} &\multicolumn{3}{|c|}{Class of} &  \multicolumn{3}{c|}{} \\
\multicolumn{3}{|c|}{thousands} &\multicolumn{3}{|c|}{units} &  \multicolumn{3}{c|}{} \\
\hline
H & T & U &H & T & U, & T\up{th} & \textbf{H\up{th}} & K\up{th} \\
\hline
\hline
 & & & 3 & 1 & \textbf{9}, & 5 & 6 & \\
\hline
\end{tabular}
\end{center}

\scriptsize
\begin{center}
% use packages: array
\begin{tabular}{ll}
H : Hundreds & T\up{th} : Tenths\\
T : Tens & H\up{th} :Hundredths\\
U : Units & K\up{e} : Thousandths\\
\end{tabular}
\end{center}

\normalsize
The answer is d).\\

7351-- Which digit is 0 in the number 125 410?\\

a) hundreds digit\\
b) hundredths digit\\
c) tenths digit\\
d) units digit\\

Answer : d)\\

Feedback :\\
\begin{center}
% use packages: array
\begin{tabular}{|rrr|rrr|rrr|}
\hline
\multicolumn{6}{|c|}{integers} &\multicolumn{3}{|c|}{decimals} \\
\hline
\multicolumn{3}{|c|}{Class of} &\multicolumn{3}{|c|}{Class of} &  \multicolumn{3}{c|}{} \\
\multicolumn{3}{|c|}{thousands} &\multicolumn{3}{|c|}{units} &  \multicolumn{3}{c|}{} \\
\hline
H & T & U &H & T & U, & T\up{th} & \textbf{H\up{th}} & K\up{th} \\
\hline
\hline
 1 & 2 & 5 & 4 & 1 & \textbf{0} & & &  \\
\hline
\end{tabular}
\end{center}

\scriptsize
\begin{center}
% use packages: array
\begin{tabular}{ll}
H : Hundreds & T\up{th} : Tenths\\
T : Tens & H\up{th} :Hundredths\\
U : Units & K\up{e} : Thousandths\\
\end{tabular}
\end{center}

\normalsize
The answer is d).\\

7352-- Which digit is 3 in the number 132,06?\\

a) hundreds digit\\
b) hundredths digit\\
c) tens digit\\
d) units digit\\

Answer : c)\\

Feedback :\\
\begin{center}
% use packages: array
\begin{tabular}{|rrr|rrr|rrr|}
\hline
\multicolumn{6}{|c|}{integers} &\multicolumn{3}{|c|}{decimals} \\
\hline
\multicolumn{3}{|c|}{Class of} &\multicolumn{3}{|c|}{Class of} &  \multicolumn{3}{c|}{} \\
\multicolumn{3}{|c|}{thousands} &\multicolumn{3}{|c|}{units} &  \multicolumn{3}{c|}{} \\
\hline
H & T & U &H & T & U, & T\up{th} & \textbf{H\up{th}} & K\up{th} \\
\hline
\hline
 & & &1 & \textbf{3} & 2, & 0 & 6 & \\
\hline
\end{tabular}
\end{center}

\scriptsize
\begin{center}
% use packages: array
\begin{tabular}{ll}
H : Hundreds & T\up{th} : Tenths\\
T : Tens & H\up{th} :Hundredths\\
U : Units & K\up{e} : Thousandths\\
\end{tabular}
\end{center}

\normalsize
The answer is c).\\



7353-- Which digit is  1 in the number 409 917?\\

a) hundreds digit\\
b) tens digit\\
c) units digit\\
d) thousands digit\\

Answer : b)\\

Feedback :\\
\begin{center}
% use packages: array
\begin{tabular}{|rrr|rrr|rrr|}
\hline
\multicolumn{6}{|c|}{integers} &\multicolumn{3}{|c|}{decimals} \\
\hline
\multicolumn{3}{|c|}{Class of} &\multicolumn{3}{|c|}{Class of} &  \multicolumn{3}{c|}{} \\
\multicolumn{3}{|c|}{thousands} &\multicolumn{3}{|c|}{units} &  \multicolumn{3}{c|}{} \\
\hline
H & T & U &H & T & U, & T\up{th} & \textbf{H\up{th}} & K\up{th} \\
\hline
\hline
4 & 0 & 9 & 9 & \textbf{1} & 7 & & & \\
\hline
\end{tabular}
\end{center}

\scriptsize
\begin{center}
% use packages: array
\begin{tabular}{ll}
H : Hundreds & T\up{th} : Tenths\\
T : Tens & H\up{th} :Hundredths\\
U : Units & K\up{e} : Thousandths\\
\end{tabular}
\end{center}

\normalsize
The answer is b).\\





7354-- Which digit is  3 in the number 309,49?\\

a) hundreds digit\\
b) tens digit\\
c) tenths digit\\
d) units digit\\

Answer : a)\\

Feedback :\\
\begin{center}
% use packages: array
\begin{tabular}{|rrr|rrr|rrr|}
\hline
\multicolumn{6}{|c|}{integers} &\multicolumn{3}{|c|}{decimals} \\
\hline
\multicolumn{3}{|c|}{Class of} &\multicolumn{3}{|c|}{Class of} &  \multicolumn{3}{c|}{} \\
\multicolumn{3}{|c|}{thousands} &\multicolumn{3}{|c|}{units} &  \multicolumn{3}{c|}{} \\
\hline
H & T & U &H & T & U, & T\up{th} & \textbf{H\up{th}} & K\up{th} \\
\hline
\hline
& & &\textbf{3} & 0 & 9, & 4 & 9 & \\
\hline
\end{tabular}
\end{center}

\scriptsize
\begin{center}
% use packages: array
\begin{tabular}{ll}
H : Hundreds & T\up{th} : Tenths\\
T : Tens & H\up{th} :Hundredths\\
U : Units & K\up{e} : Thousandths\\
\end{tabular}
\end{center}

\normalsize
The answer is a).\\





7355-- Which digit is 0 in the number 14 075?\\

a) hundreds digit\\
b) tens digit\\
c) tens of thousands digit\\
d) thousands digit\\

Answer: a)\\

Feedback :\\
\begin{center}
% use packages: array
\begin{tabular}{|rrr|rrr|rrr|}
\hline
\multicolumn{6}{|c|}{integers} &\multicolumn{3}{|c|}{decimals} \\
\hline
\multicolumn{3}{|c|}{Class of} &\multicolumn{3}{|c|}{Class of} &  \multicolumn{3}{c|}{} \\
\multicolumn{3}{|c|}{thousands} &\multicolumn{3}{|c|}{units} &  \multicolumn{3}{c|}{} \\
\hline
H & T & U &H & T & U, & T\up{th} & \textbf{H\up{th}} & K\up{th} \\
\hline
\hline
 & 1 & 4 & \textbf{0} & 7 & 5 & & & \\
\hline
\end{tabular}
\end{center}

\scriptsize
\begin{center}
% use packages: array
\begin{tabular}{ll}
H : Hundreds & T\up{th} : Tenths\\
T : Tens & H\up{th} :Hundredths\\
U : Units & K\up{e} : Thousandths\\
\end{tabular}
\end{center}

\normalsize
The answer is a).\\






7356-- Which digit is 9 in the numbere 129 758?\\

a) hundreds of thousands digit\\
b) tens digit\\
c) tens of thousands digit\\
d) thousands digit\\

Answer : d)\\

Feedback :\\
\begin{center}
% use packages: array
\begin{tabular}{|rrr|rrr|rrr|}
\hline
\multicolumn{6}{|c|}{integers} &\multicolumn{3}{|c|}{decimals} \\
\hline
\multicolumn{3}{|c|}{Class of} &\multicolumn{3}{|c|}{Class of} &  \multicolumn{3}{c|}{} \\
\multicolumn{3}{|c|}{thousands} &\multicolumn{3}{|c|}{units} &  \multicolumn{3}{c|}{} \\
\hline
H & T & U &H & T & U, & T\up{th} & \textbf{H\up{th}} & K\up{th} \\
\hline
\hline
 1 & 2 & \textbf{9} & 7 & 5 & 8 & & & \\
\hline
\end{tabular}
\end{center}

\scriptsize
\begin{center}
% use packages: array
\begin{tabular}{ll}
H : Hundreds & T\up{th} : Tenths\\
T : Tens & H\up{th} :Hundredths\\
U : Units & K\up{e} : Thousandths\\
\end{tabular}
\end{center}

\normalsize
The answer is d).\\






7357-- Which digit is 1 in the number 1025?\\

a) hundreds digit\\
b) tens digit\\
c) units digit\\
d) thousands digit\\

Answer : d)\\

Feedback :\\
\begin{center}
% use packages: array
\begin{tabular}{|rrr|rrr|rrr|}
\hline
\multicolumn{6}{|c|}{integers} &\multicolumn{3}{|c|}{decimals} \\
\hline
\multicolumn{3}{|c|}{Class of} &\multicolumn{3}{|c|}{Class of} &  \multicolumn{3}{c|}{} \\
\multicolumn{3}{|c|}{thousands} &\multicolumn{3}{|c|}{units} &  \multicolumn{3}{c|}{} \\
\hline
H & T & U &H & T & U, & T\up{th} & \textbf{H\up{th}} & K\up{th} \\
\hline
\hline
  &  & \textbf{1} & 0 & 2 & 5 & & & \\
\hline
\end{tabular}
\end{center}

\scriptsize
\begin{center}
% use packages: array
\begin{tabular}{ll}
H : Hundreds & T\up{th} : Tenths\\
T : Tens & H\up{th} :Hundredths\\
U : Units & K\up{e} : Thousandths\\
\end{tabular}
\end{center}

\normalsize
The answer is d).\\





7358-- Which digit 5 in the numbere 950 000?\\

a) hundreds digit\\
b) tens digit\\
c) tens of thousands digit\\
d) thousands digit\\

Answer : c)\\

Feedback :\\
\begin{center}
% use packages: array
\begin{tabular}{|rrr|rrr|rrr|}
\hline
\multicolumn{6}{|c|}{integers} &\multicolumn{3}{|c|}{decimals} \\
\hline
\multicolumn{3}{|c|}{Class of} &\multicolumn{3}{|c|}{Class of} &  \multicolumn{3}{c|}{} \\
\multicolumn{3}{|c|}{thousands} &\multicolumn{3}{|c|}{units} &  \multicolumn{3}{c|}{} \\
\hline
H & T & U &H & T & U, & T\up{th} & \textbf{H\up{th}} & K\up{th} \\
\hline
\hline
 9 & \textbf{5} & 0 & 0 & 0 & 0 & & &\\
\hline
\end{tabular}
\end{center}

\scriptsize
\begin{center}
% use packages: array
\begin{tabular}{ll}
H : Hundreds & T\up{th} : Tenths\\
T : Tens & H\up{th} :Hundredths\\
U : Units & K\up{e} : Thousandths\\
\end{tabular}
\end{center}

\normalsize
The answer is c).\\





7359-- Which digit is 4 in the number 42 100?\\

a) hundreds digit\\
b) tens digit\\
c) tens of thousands digit\\
d) thousands digit\\

Answer : c)\\

Feedback :\\
\begin{center}
% use packages: array
\begin{tabular}{|rrr|rrr|rrr|}
\hline
\multicolumn{6}{|c|}{integers} &\multicolumn{3}{|c|}{decimals} \\
\hline
\multicolumn{3}{|c|}{Class of} &\multicolumn{3}{|c|}{Class of} &  \multicolumn{3}{c|}{} \\
\multicolumn{3}{|c|}{thousands} &\multicolumn{3}{|c|}{units} &  \multicolumn{3}{c|}{} \\
\hline
H & T & U &H & T & U, & T\up{th} & \textbf{H\up{th}} & K\up{th} \\
\hline
\hline
  & \textbf{4} & 2 & 1 & 0 & 0 & & & \\
\hline
\end{tabular}
\end{center}

\scriptsize
\begin{center}
% use packages: array
\begin{tabular}{ll}
H : Hundreds & T\up{th} : Tenths\\
T : Tens & H\up{th} :Hundredths\\
U : Units & K\up{e} : Thousandths\\
\end{tabular}
\end{center}

\normalsize
The answer is c).\\





7360-- Which digit is 1 in the number 100 300?\\

a) hundreds digit\\
b) hundreds of thousands digit\\
c) tens digit\\
d) units digit\\

Answer : b)\\

Feedback :\\
\begin{center}
% use packages: array
\begin{tabular}{|rrr|rrr|rrr|}
\hline
\multicolumn{6}{|c|}{integers} &\multicolumn{3}{|c|}{decimals} \\
\hline
\multicolumn{3}{|c|}{Class of} &\multicolumn{3}{|c|}{Class of} &  \multicolumn{3}{c|}{} \\
\multicolumn{3}{|c|}{thousands} &\multicolumn{3}{|c|}{units} &  \multicolumn{3}{c|}{} \\
\hline
H & T & U &H & T & U, & T\up{th} & \textbf{H\up{th}} & K\up{th} \\
\hline
\hline
 \textbf{1} & 0 & 0 & 3 & 0 & 0 & & & \\
\hline
\end{tabular}
\end{center}

\scriptsize
\begin{center}
% use packages: array
\begin{tabular}{ll}
H : Hundreds & T\up{th} : Tenths\\
T : Tens & H\up{th} :Hundredths\\
U : Units & K\up{e} : Thousandths\\
\end{tabular}
\end{center}

\normalsize
The answer is b).\\



7361-- Which digit is 6 in the number 632 900?\\

a) hundreds of thousands digit\\
b) tens of thousands digit\\
c) units digit\\
d) thousands digit\\

Answer : a)\\

Feedback :\\
\begin{center}
% use packages: array
\begin{tabular}{|rrr|rrr|rrr|}
\hline
\multicolumn{6}{|c|}{integers} &\multicolumn{3}{|c|}{decimals} \\
\hline
\multicolumn{3}{|c|}{Class of} &\multicolumn{3}{|c|}{Class of} &  \multicolumn{3}{c|}{} \\
\multicolumn{3}{|c|}{thousands} &\multicolumn{3}{|c|}{units} &  \multicolumn{3}{c|}{} \\
\hline
H & T & U &H & T & U, & T\up{th} & \textbf{H\up{th}} & K\up{th} \\
\hline
\hline
 \textbf{6} & 3 & 2 & 9 & 0 & 0 & & & \\
\hline
\end{tabular}
\end{center}

\scriptsize
\begin{center}
% use packages: array
\begin{tabular}{ll}
H : Hundreds & T\up{th} : Tenths\\
T : Tens & H\up{th} :Hundredths\\
U : Units & K\up{e} : Thousandths\\
\end{tabular}
\end{center}

\normalsize
The answer is a).\\


7362-- Put the numbers in ascending order.\\  
\begin{center}
10,90\ \ \ 10,80\ \ \ 11,01\\
\end{center}

a) 10,80\ \ \ 10,90\ \ \ 11,01\\
b) 10,80\ \ \ 11,01\ \ \ 10,90\\
c) 10,90\ \ \ 11,01\ \ \ 10,80\\
d) 11,01\ \ \ 10,90\ \ \ 10,80\\

Answer : a)\\

Feedback :\\
To put the numbers in ascending order means ordering them from lowest to highest.\\
\begin{center}
% use packages: array
\begin{tabular}{|rrr|rrr|rrr|}
\hline
\multicolumn{6}{|c|}{integers} &\multicolumn{3}{|c|}{decimals} \\
\hline
\multicolumn{3}{|c|}{Class of} &\multicolumn{3}{|c|}{Class of} &  \multicolumn{3}{c|}{} \\
\multicolumn{3}{|c|}{thousands} &\multicolumn{3}{|c|}{units} &  \multicolumn{3}{c|}{} \\
\hline
H & T & U &H & T & U, & T\up{th} & \textbf{H\up{th}} & K\up{th} \\
\hline
\hline
& & & & 1 & 0, & 8 & 0 &\\
& & & & 1 & 0, & 9 & 0 &\\
& & & & 1 & 1, & 0 & 1 &\\
\hline
\end{tabular}
\end{center}

\scriptsize
\begin{center}
% use packages: array
\begin{tabular}{ll}
H : Hundreds & T\up{th} : Tenths\\
T : Tens & H\up{th} :Hundredths\\
U : Units & K\up{e} : Thousandths\\
\end{tabular}
\end{center}

\normalsize
The answer is a).\\


7363-- Put the numbers in ascending order.\\ 
\begin{center}
1,89\ \ \ 1,90\ \ \ 1,88\\
\end{center}

a) 1,88\ \ \ 1,89\ \ \ 1,90\\
b) 1,88\ \ \ 1,90\ \ \ 1,89\\
c) 1,89\ \ \ 1,90\ \ \ 1,88\\
d) 1,90\ \ \ 1,89\ \ \ 1,88\\

Answer : a)\\

Feedaback :\\
To put the numbers in ascending order means ordering them from lowest to highest.\\
\begin{center}
% use packages: array
\begin{tabular}{|rrr|rrr|rrr|}
\hline
\multicolumn{6}{|c|}{integers} &\multicolumn{3}{|c|}{decimals} \\
\hline
\multicolumn{3}{|c|}{Class of} &\multicolumn{3}{|c|}{Class of} &  \multicolumn{3}{c|}{} \\
\multicolumn{3}{|c|}{thousands} &\multicolumn{3}{|c|}{units} &  \multicolumn{3}{c|}{} \\
\hline
H & T & U &H & T & U, & T\up{th} & \textbf{H\up{th}} & K\up{th} \\
\hline
\hline
& & & &  & 1, & 8 & 8 &\\
& & & &  & 1, & 8 & 9 &\\
& & & &  & 1, & 9 & 0 &\\
\hline
\end{tabular}
\end{center}

\scriptsize
\begin{center}
% use packages: array
\begin{tabular}{ll}
H : Hundreds & T\up{th} : Tenths\\
T : Tens & H\up{th} :Hundredths\\
U : Units & K\up{e} : Thousandths\\
\end{tabular}
\end{center}

\normalsize
The answer is a).\\


7364--Put the numbers in ascending order.\\ 
\begin{center}
985 541, 309 405, 196 000\\
\end{center}

a) 196 000, 309 405, 985 541\\
b) 309 405, 196 000, 985 541\\
c) 309 405, 985 541, 196 000\\
d) 985 541, 196 000, 309 405\\

Answer : a)\\

Feedback :\\
To put the numbers in ascending order means ordering them from lowest to highest.\\
\begin{center}
% use packages: array
\begin{tabular}{|rrr|rrr|rrr|}
\hline
\multicolumn{6}{|c|}{integers} &\multicolumn{3}{|c|}{decimals} \\
\hline
\multicolumn{3}{|c|}{Class of} &\multicolumn{3}{|c|}{Class of} &  \multicolumn{3}{c|}{} \\
\multicolumn{3}{|c|}{thousands} &\multicolumn{3}{|c|}{units} &  \multicolumn{3}{c|}{} \\
\hline
H & T & U &H & T & U, & T\up{th} & \textbf{H\up{th}} & K\up{th} \\
\hline
\hline
1 & 9 & 6 & 0 & 0 & 0 & & &\\
3 & 0 & 9 & 4 & 0 & 5 & & &\\
9 & 8 & 5 & 5 & 4 & 1 & & &\\
\hline
\end{tabular}
\end{center}

\scriptsize
\begin{center}
% use packages: array
\begin{tabular}{ll}
H : Hundreds & T\up{th} : Tenths\\
T : Tens & H\up{th} :Hundredths\\
U : Units & K\up{e} : Thousandths\\
\end{tabular}
\end{center}

\normalsize
The answer is a).\\


7365-- Put the numbers in decreasing order.\\ 
\begin{center}
5,55\ \ \ 6,54\ \ \ 5,56\\
\end{center}

a) 5,55\ \ \ 5,56\ \ \ 6,54\\
b) 5,56\ \ \ 5,55\ \ \ 6,54\\
c) 6,54\ \ \ 5,55\ \ \ 5,56\\
d) 6,54\ \ \ 5,56\ \ \ 5,55\\

Answer : d)\\

Feedback :\\
To put the numbers in decreasing order means ordering them from highest to lowest.\\
\begin{center}
% use packages: array
\begin{tabular}{|rrr|rrr|rrr|}
\hline
\multicolumn{6}{|c|}{integers} &\multicolumn{3}{|c|}{decimals} \\
\hline
\multicolumn{3}{|c|}{Class of} &\multicolumn{3}{|c|}{Class of} &  \multicolumn{3}{c|}{} \\
\multicolumn{3}{|c|}{thousands} &\multicolumn{3}{|c|}{units} &  \multicolumn{3}{c|}{} \\
\hline
H & T & U &H & T & U, & T\up{th} & \textbf{H\up{th}} & K\up{th} \\
\hline
\hline
& & &  &  & 6, & 5 & 4 &\\
& & &  &  & 5, & 5 & 6 &\\
& & &  &  & 5, & 5 & 5 &\\
\hline
\end{tabular}
\end{center}

\scriptsize
\begin{center}
% use packages: array
\begin{tabular}{ll}
H : Hundreds & T\up{th} : Tenths\\
T : Tens & H\up{th} :Hundredths\\
U : Units & K\up{e} : Thousandths\\
\end{tabular}
\end{center}

\normalsize
The answer is d).\\


7366-- Put the numbers in decreasing order.\\ 
\begin{center}
196 541, 197 301, 231 200.\\
\end{center}

a) 196 541, 197 301, 231 200\\
b) 197 301, 231 200, 196 541\\
c) 231 200, 196 541, 197 301\\
d) 231 200, 197 301, 196 541\\

Answer : d)\\

Feedback :\\
To put the numbers in decreasing order means ordering them from highest to lowest.\\
\begin{center}
% use packages: array
\begin{tabular}{|rrr|rrr|rrr|}
\hline
\multicolumn{6}{|c|}{integers} &\multicolumn{3}{|c|}{decimals} \\
\hline
\multicolumn{3}{|c|}{Class of} &\multicolumn{3}{|c|}{Class of} &  \multicolumn{3}{c|}{} \\
\multicolumn{3}{|c|}{thousands} &\multicolumn{3}{|c|}{units} &  \multicolumn{3}{c|}{} \\
\hline
H & T & U &H & T & U, & T\up{th} & \textbf{H\up{th}} & K\up{th} \\
\hline
\hline
 2& 3 & 1 & 2 & 0 & 0 & & &\\
 1& 9 & 7 & 3 & 0 & 1 & & &\\
 1& 9 & 6 & 5 & 4 & 1 & & &\\
\hline
\end{tabular}
\end{center}

\scriptsize
\begin{center}
% use packages: array
\begin{tabular}{ll}
H : Hundreds & T\up{th} : Tenths\\
T : Tens & H\up{th} :Hundredths\\
U : Units & K\up{e} : Thousandths\\
\end{tabular}
\end{center}

\normalsize
The answer is d).\\


7367-- Choose the circular diagram that corresponds to the statement.\\ 
\begin{center}
In Mrs Annie's class, two students out of five have a cat at home.\\
\end{center}

a)\\
\includegraphics[width=2cm]{Q7367a.eps}
% Q7367a.eps : 300dpi, width=3.39cm, height=3.39cm, bb=0 0 400 400
\\

b)\\
\includegraphics[width=2cm]{Q7367b.eps}
% Q7367b.eps : 300dpi, width=3.39cm, height=3.39cm, bb=0 0 400 400
\\

c)\\
\includegraphics[width=2cm]{Q7367c.eps}
% Q7367c.eps : 300dpi, width=3.39cm, height=3.39cm, bb=0 0 400 400
\\

d)\\
\includegraphics[width=2cm]{Q7367d.eps}
% Q7367d.eps : 300dpi, width=3.39cm, height=3.39cm, bb=0 0 400 400
\\

Answer : b)\\

Feedback :\\
\begin{center}
\includegraphics[width=5cm]{R7367.eps}
% R7367.eps : 300dpi, width=3.39cm, height=3.39cm, bb=0 0 400 400
\end{center}
The answer is b).\\

7368-- Complete the sentence.
\begin{center}
Any number is divisible by two without remainder if the units digit is evenly divisible by \underline{\quad\quad}.\\
\end{center}
 
a) two\\
b) four\\
c) three\\
d) one\\

Answer: a)\\

Feedback : \\
\begin{center}
Any number is divisable by two without remainder if the units 
digit is evenly divisable by \textbf{deux}.\\
\end{center}

Exemple : 986 556\\
\begin{eqnarray*}
\frac{6}{2}= 3
\end{eqnarray*}
\begin{eqnarray*}
\frac{986\ 556}{2}= 493\ 278
\end{eqnarray*}


The answer is a).\\

7369-- Complete the following sentence.
\begin{center}
Any number is divisable by three without remainder if the sum of its digits is evenly divisable by \underline{\quad\quad}.\\
\end{center}

a) two\\
b) four\\
c) three\\
d) one\\

Answer : c)\\

Feedback :
\begin{center}
Any number is divisable by three without remainder if the sum of its digits is evenly divisable by \textbf{trois}.\\
\end{center}

Example : 986 556\\
\begin{eqnarray*}
9+8+6+5+5+6 = 39
\end{eqnarray*}
\begin{eqnarray*}
\frac{39}{3}=13
\end{eqnarray*}
\begin{eqnarray*}
\frac{986\ 556}{3}= 328\ 852
\end{eqnarray*}

The answer is c).\\


\end{document}
