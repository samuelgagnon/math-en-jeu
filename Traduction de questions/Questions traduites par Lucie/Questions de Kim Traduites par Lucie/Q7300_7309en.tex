\documentclass[letterpaper, 12pt]{article}
\usepackage[french]{babel}

\usepackage{amsmath,amsfonts,amsthm,amssymb,graphicx,multirow,hyperref,color}
\usepackage[latin1]{inputenc}

\pagestyle{plain}

\setlength{\topmargin}{-2cm}
\setlength{\textheight}{23.5cm}
\setlength{\textwidth}{12cm}
\setlength{\oddsidemargin}{-1cm}
\setlength{\parindent}{0pt}

\begin{document}


7300-- Which of the following is an even number?\\

a) 125 648\\
b) 385 333\\
c) 568 369\\
d) 777 777\\

Answer : a)\\

Feedback :\\
The number 125 648 is evenly divisable by 2, because the 
units digit is divisable by 2 without remainder.\\
\begin{eqnarray*}
\frac{8}{2}=4\\
\end{eqnarray*}
Since 125 648 is evenly divisable by 2, it is an even number.\\
The answer is a).\\



7301-- Which of the following is an odd number?\\

a) 5648\\
b) 85 334\\
c) 568 362\\
d) 777 777\\

Answer : d)\\

Feedback :\\
The number 777 777 is not evenly divisable by 2, because the units 
digit is not divisable by 2 without remainder.\\
\begin{eqnarray*}
\frac{7}{2}=3,5\\
\end{eqnarray*}
Since 777 777 is not evenly divisable by 2, it is an odd number.\\
The answer is d).\\




7302-- Which of the following is an even number?\\

a) 1641\\
b) 5333\\
c) 854 532\\
d) 999 997\\

Answer : c)\\

Feedback :\\
The number 854 532 is evenly divisable by 2, because the units digit is divisable by 2 without remainder.\\
\begin{eqnarray*}
\frac{2}{2}=1\\
\end{eqnarray*}
Since 854 532 is evenly divisable by 2, it is an even number.\\
The answer is c).\\




7303-- Which of the following is an odd number?\\

a) 128 566\\
b) 333 334\\
c) 659 999\\
d) 777 772\\

Answer : c)\\

Feedback :\\
The number 659 999 is not evenly divisable by 2, because the 
units digit is not divisable by 2 without remainder.\\
\begin{eqnarray*}
\frac{9}{2}=4,5\\
\end{eqnarray*}
Since 659 999 is not evenly divisable by 2, it is an odd number.\\
The answer is c).\\

7304-- Complete the following sentence.\\
\begin{center}
Any number is divisable by four without remainder if the number 
created by the two last digits is evenly divisable by  \underline{\quad\quad}.\\
\end{center}

a) Ten\\
b) Eleven\\
c) Four\\
d) One\\

Answer : c)\\

Feedback:\\
\begin{center}
Any number is divisable by four without remainder if the number created by the two last digits is evenly divisable by \textbf{four}.\\
\end{center}

Exemple: 7316
\begin{eqnarray*}
\frac{16}{4}=4\\
\end{eqnarray*}
\begin{eqnarray*}
\frac{7316}{4}=1829\\
\end{eqnarray*}
The answer is c).\\



7305-- In the number 123.45, which digit is number 5?\\

a) hundredths digit\\
b) tenths digit\\
c) hundreds digit\\
d) units digit\\

Answer : a)\\

Feedback :\\
\begin{center}
% use packages: array
\begin{tabular}{|rrr|rrr|rrr|}
\hline
\multicolumn{6}{|c|}{entiers} &\multicolumn{3}{|c|}{d\'ecimaux} \\
\hline
\multicolumn{3}{|c|}{classe des} &\multicolumn{3}{|c|}{classe des} &  \multicolumn{3}{c|}{} \\
\multicolumn{3}{|c|}{mille} &\multicolumn{3}{|c|}{unit\'es} &  \multicolumn{3}{c|}{} \\
\hline
C & D & U &C & D & U, & D\up{e} & \textbf{C\up{e}} & M\up{e} \\
\hline
\hline
&  &  & 1 & 2 & 3, & 4 & \textbf{5} &  \\
\hline
\end{tabular}
\end{center}

\scriptsize
\begin{center}
% use packages: array
\begin{tabular}{ll}
C : centaine & D\up{e} : dixi�me\\
D : dizaine & C\up{e} : centi�me\\
U : unit� & M\up{e} : milli�me\\
\end{tabular}
\end{center}

\normalsize

The answer is a).\\




7306-- Express the following multiplication in terms of powers.\\ 
\begin{center}
$5\times5\times5$\\.
\end{center}

a) $ 2^{1}$\\
b) $ 3^{3}$\\
c) $ 5^{3}$\\
d) $ 5^{4}$\\

Answer : c)\\

Feedback :\\
$ 5^{3}$ = $5\times5\times5$\\
The answer is c).\\




7307-- Express the following multiplication in terms of powers.\\ 
\begin{center}
$7\times4\times7\times4$\\
\end{center}

a) $ 4^{4}$\\
b) $ 7^{4}$\\
c) $ 4^{2}\times7^{2}$\\
d) $ 10^{5}\times10^{8}$\\

Answer : c)\\

Feedback :\\
First of all, because of the commutativity of multiplications, the same numbers are put together. 
Afterwards, only one number from each group is kept and the powers are added.

\begin{eqnarray*}
7\times4\times7\times4
&=&4\times4\times7\times7\\
&=&4^{2}\times7^{2}\\                                                                           \end{eqnarray*}
The answer is c).\\




7308-- Which number is represented by the addition?\\ 
\begin{center}
800 000 + 70 000 + 500 + 20 + 3\\
\end{center}

a) 32 578\\
b) 78 523\\
c) 87 523\\
d) 870 523\\

Answer : d)\\

Feedback :\\
\begin{center}
% use packages: array
\begin{tabular}{|rrr|rrr|rrr|}
\hline
\multicolumn{6}{|c|}{entiers} &\multicolumn{3}{|c|}{d\'ecimaux} \\
\hline
\multicolumn{3}{|c|}{classe des} &\multicolumn{3}{|c|}{classe des} &  \multicolumn{3}{c|}{} \\
\multicolumn{3}{|c|}{mille} &\multicolumn{3}{|c|}{unit\'es} &  \multicolumn{3}{c|}{} \\
\hline
C & D & U &C & D & U, & D\up{e} & C\up{e} & M\up{e} \\
\hline
\hline
8 & 0 & 0 & 0 & 0 & 0 &  & &\\
 & 7 & 0 & 0 & 0 & 0 &  & &\\
+ &  &  & 5 & 0 & 0 &  & &\\
 &  &  &  & 2 & 0 &  & &\\
 &  &  &  &  & 3 &  & &\\
\hline
\hline
 8 & 7 & 0 & 5 & 2 & 3 &  & &
\\
\hline
\end{tabular}
\end{center}

\scriptsize
\begin{center}
% use packages: array
\begin{tabular}{ll}
C : centaine & D\up{e} : dixi�me\\
D : dizaine & C\up{e} : centi�me\\
U : unit� & M\up{e} : milli�me\\
\end{tabular}
\end{center}

\normalsize
The answer is d).\\




7309-- Sort in ascending order.\\ 
\begin{center}
101, 8997, 10 722\\
\end{center}

a) 101, 8997, 10 722\\
b) 101, 10 722, 8997\\
c) 8997, 10 722, 101\\
d) 10 722, 101, 8997\\\\

Answer : a)\\

Feedback :\\
\begin{center}
% use packages: array
\begin{tabular}{|rrr|rrr|rrr|}
\hline
\multicolumn{6}{|c|}{entiers} &\multicolumn{3}{|c|}{d\'ecimaux} \\
\hline
\multicolumn{3}{|c|}{classe des} &\multicolumn{3}{|c|}{classe des} &  \multicolumn{3}{c|}{} \\
\multicolumn{3}{|c|}{mille} &\multicolumn{3}{|c|}{unit\'es} &  \multicolumn{3}{c|}{} \\
\hline
C & D & U &C & D & U, & D\up{e} & C\up{e} & M\up{e} \\
\hline
\hline
 &  &  & 1 & 0 & 1 &  &  & \\
 &  & 8 & 9 & 9 & 7 &  &  &\\
 & 1 & 0 & 7 & 2 & 2 &  &  &\\
\hline
\end{tabular}
\end{center}

\scriptsize
\begin{center}
% use packages: array
\begin{tabular}{ll}
C : centaine & D\up{e} : dixi�me\\
D : dizaine & C\up{e} : centi�me\\
U : unit� & M\up{e} : milli�me\\
\end{tabular}
\end{center}

\normalsize

The answer is a).\\






\end{document}
