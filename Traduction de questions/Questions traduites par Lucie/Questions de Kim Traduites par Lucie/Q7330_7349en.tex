\documentclass[letterpaper, 12pt]{article}
\usepackage[french]{babel}

\usepackage{amsmath,amsfonts,amsthm,amssymb,graphicx,multirow,hyperref,color}
\usepackage[latin1]{inputenc}

\pagestyle{plain}

\setlength{\topmargin}{-2cm}
\setlength{\textheight}{23.5cm}
\setlength{\textwidth}{12cm}
\setlength{\oddsidemargin}{-1cm}
\setlength{\parindent}{0pt}

\begin{document}

7330-- Which portion of the figure is colored?\\

\begin{center}
\includegraphics[width=5cm]{Q7330.eps}
% Q7330.eps : 300dpi, width=3.39cm, height=3.39cm, bb=0 0 400 400
\end{center}

a) $\frac{1}{4}$\\

b) $\frac{1}{2}$\\

c) $\frac{2}{3}$\\

d) 1\\

Answer : b)\\

Feedback :\\
First, the figure is a rectangle separated into two right triangles by its diagonal. Next, there is one triangle out of two that is colored. Therefore, the colored portion is $\frac{1}{2}$.\\
The answer is b).\\


7331--What is the sum of 3587,29 + 0,10?\\

a) 3587,39\\
b) 3588,29\\
c) 3597,39\\
d) 4587,29\\

Answer : a)\\

Feedback :\\
\begin{center}
% use packages: array
\begin{tabular}{|rrr|rrr|rrr|}
\hline
\multicolumn{6}{|c|}{integers} &\multicolumn{3}{|c|}{decimals} \\
\hline
\multicolumn{3}{|c|}{Class of} &\multicolumn{3}{|c|}{Class of} &  \multicolumn{3}{c|}{} \\
\multicolumn{3}{|c|}{thousands} &\multicolumn{3}{|c|}{units} &  \multicolumn{3}{c|}{} \\
\hline
H & T & U &H & T & U, & T\up{th} & \textbf{H\up{th}} & K\up{th} \\
\hline
\hline
&  & 3 & 5 & 8 & 7, & 2 & 9 &  \\
 & + &  &  &  & 0, & 1 & 0 &  \\
\hline
\hline
 &  & 3 & 5 & 8 & 7, & 3 & 9 &  \\
\hline
\end{tabular}
\end{center}

\scriptsize
\begin{center}
% use packages: array
\begin{tabular}{ll}
H : Hundreds & T\up{th} : Tenths\\
T : Tens & H\up{th} :Hundredths\\
U : Units & K\up{e} : Thousandths\\
\end{tabular}
\end{center}

\normalsize

The answer is a).\\


7332-- Today, the outside temperature is of $-5^{o}$C. If we predict a drop of $5^{o}$C for tomorrow, what will the temperature \mbox{be}\\

a) $-10^{o}$C\\
b) $-5^{o}$C\\
c) $0^{o}$C\\
d) $5^{o}$C\\

Answer : a)\\

Feedback :\\
\begin{center}
\includegraphics[width=5cm]{Q7332.eps}
% Q7332.eps : 300dpi, width=3.39cm, height=3.39cm, bb=0 0 400 400
\end{center}

We withdraw 5 to the number $- 5$.\\
\begin{eqnarray*}
-5 - 5 = - 10\\
\end{eqnarray*}

The answer is a).\\





7333-- Which fraction of the set is represented by the suns?

\begin{center}
    \includegraphics[width=6cm]{Q7333.eps}
% Q7333.eps : 300dpi, width=3.39cm, height=3.39cm, bb=0 0 400 400
    \end{center}


a) $\frac{1}{3}$\\\\
b) $\frac{1}{2}$\\\\
c) $\frac{4}{6}$\\\\
d) $\frac{6}{3}$\\\\

Answer : b)\\

Feedback :\\
On a set of six items, there are three suns.\\
\begin{eqnarray*}
\frac{3}{6}\\
\end{eqnarray*}

If we reduce $\frac{3}{6}$ to its simplest form, we get $\frac{1}{2}$.\\

\begin{center}
\begin{tabular}{c}

$\div 3$  \\
\LARGE $\curvearrowright$  \\
\Large $\frac{3}{6}$  =  \Large $\frac{1}{2}$ \\
\rotatebox{180}{\LARGE$\curvearrowleft$}\\
$\div 3$  \\

\end{tabular}
\end{center}
The answer is  b).\\




7334-- Considering the order of operation, by which operation do we have to start?\\
\begin{eqnarray*}
20+10\times3\div1-4\\
\end{eqnarray*}


a) $1-4$\\
b) $3\div1$\\
c) $10\times3$\\
d) $20+10$\\

Answer : c)\\

Feedback :\\
First, to determine the order of operations, we need to look from left to right. Then, we have to start by the first multiplication or division we come accross.\\
\begin{eqnarray*}
10\times3\\
\end{eqnarray*}
The answer is c).\\



7335-- Considering the order of operations, by which operation should we start?\\

\begin{eqnarray*}
20+10\times3\div(1-4)\\
\end{eqnarray*}

a) $1-4$\\
b) $3\div1$\\
c) $10\times3$\\
d) $20+10$\\

Answer : a)\\

Feedback :\\
We have to start by the operation in parentheses.\\
\begin{eqnarray*}
1-4.
\end{eqnarray*}
The answer is a).\\

7336-- Which number is represented by the addition?\\
\begin{center}
100 000 + 20 000 + 1000 + 200 + 10 + 2
\end{center}

a) 12 221\\
b) 121 212\\
c) 300 000\\
d) 429 200\\

Answer : b)\\

Feedback :\\
\begin{center}
% use packages: array
\begin{tabular}{|rrr|rrr|rrr|}
\hline
\multicolumn{6}{|c|}{integers} &\multicolumn{3}{|c|}{decimals} \\
\hline
\multicolumn{3}{|c|}{Class of} &\multicolumn{3}{|c|}{Class of} &  \multicolumn{3}{c|}{} \\
\multicolumn{3}{|c|}{thousands} &\multicolumn{3}{|c|}{units} &  \multicolumn{3}{c|}{} \\
\hline
H & T & U &H & T & U, & T\up{th} & \textbf{H\up{th}} & K\up{th} \\
\hline
\hline
1 & 0 & 0 & 0 & 0 & 0 &  &  &  \\
  & 2 & 0 & 0 & 0 & 0 &  &  &  \\
+ &   & 1 & 0 & 0 & 0 &  &  &  \\
  &   &   & 2 & 0 & 0 &  &  &  \\
  &   &   &   & 1 & 0 &  &  &  \\
  &   &   &   &   & 2 &  &  &  \\
\hline
\hline
1 & 2 & 1 & 2 & 1 & 2 &  &  &  \\
\hline
\end{tabular}
\end{center}

\scriptsize
\begin{center}
% use packages: array
\begin{tabular}{ll}
H : Hundreds & T\up{th} : Tenths\\
T : Tens & H\up{th} :Hundredths\\
U : Units & K\up{e} : Thousandths\\
\end{tabular}
\end{center}

\normalsize
The answer is b).\\



7337-- Which number is represented by the addition?\\
\begin{center}
70 000 + 1 000 + 400 + 30
\end{center}

a) 7143\\
b) 71 430\\
c) 71 433\\
d) 704 300\\

Answer : b)\\

Feedback :\\
\begin{center}
% use packages: array
\begin{tabular}{|rrr|rrr|rrr|}
\hline
\multicolumn{6}{|c|}{integers} &\multicolumn{3}{|c|}{decimals} \\
\hline
\multicolumn{3}{|c|}{Class of} &\multicolumn{3}{|c|}{Class of} &  \multicolumn{3}{c|}{} \\
\multicolumn{3}{|c|}{thousands} &\multicolumn{3}{|c|}{units} &  \multicolumn{3}{c|}{} \\
\hline
H & T & U &H & T & U, & T\up{th} & \textbf{H\up{th}} & K\up{th} \\
\hline
\hline
  & 7 & 0 & 0 & 0 & 0 &  &  & \\
  &   & 1 & 0 & 0 & 0 &  &  & \\
+ &   &   & 4 & 0 & 0 &  &  & \\
  &   &   &   & 3 & 0 &  &  & \\
\hline
\hline
  & 7 & 1 & 4 & 3 & 0 &  &  & \\
\hline
\end{tabular}
\end{center}

\scriptsize
\begin{center}
% use packages: array
\begin{tabular}{ll}
H : Hundreds & T\up{th} : Tenths\\
T : Tens & H\up{th} :Hundredths\\
U : Units & K\up{e} : Thousandths\\
\end{tabular}
\end{center}

\normalsize

The answer is b).\\


7338-- Which number is represented by the addition?\\
\begin{center}
600 000 + 60 000 + 6000 + 60 + 6
\end{center}

a) 66 666\\
b) 660 666\\
c) 666 066\\
d) 666 666\\

Answer : c)\\

Feedback :\\
\begin{center}
% use packages: array
\begin{tabular}{|rrr|rrr|rrr|}
\hline
\multicolumn{6}{|c|}{integers} &\multicolumn{3}{|c|}{decimals} \\
\hline
\multicolumn{3}{|c|}{Class of} &\multicolumn{3}{|c|}{Class of} &  \multicolumn{3}{c|}{} \\
\multicolumn{3}{|c|}{thousands} &\multicolumn{3}{|c|}{units} &  \multicolumn{3}{c|}{} \\
\hline
H & T & U &H & T & U, & T\up{th} & \textbf{H\up{th}} & K\up{th} \\
\hline
\hline
6 & 0 & 0 & 0 & 0 & 0 & & &\\
  & 6 & 0 & 0 & 0 & 0 & & &\\
+ &   & 6 & 0 & 0 & 0 & & &\\
  &   &   &   & 6 & 0 & & &\\
  &  &    &   &   & 6 & & &\\
\hline
\hline
6 & 6 & 6 & 0 & 6 & 6 & & &\\
\hline
\end{tabular}
\end{center}

\scriptsize
\begin{center}
% use packages: array
\begin{tabular}{ll}
H : Hundreds & T\up{th} : Tenths\\
T : Tens & H\up{th} :Hundredths\\
U : Units & K\up{e} : Thousandths\\
\end{tabular}
\end{center}

\normalsize

The answer is c).\\


7339-- Which number is represented by the addition?\\
\begin{center}
5 000 + 400 + 10 + 3
\end{center}

a) 314\\
b) 513\\
c) 5000\\
d) 5413\\

Answer : d)\\

Feedback :\\
\begin{center}
% use packages: array
\begin{tabular}{|rrr|rrr|rrr|}
\hline
\multicolumn{6}{|c|}{integers} &\multicolumn{3}{|c|}{decimals} \\
\hline
\multicolumn{3}{|c|}{Class of} &\multicolumn{3}{|c|}{Class of} &  \multicolumn{3}{c|}{} \\
\multicolumn{3}{|c|}{thousands} &\multicolumn{3}{|c|}{units} &  \multicolumn{3}{c|}{} \\
\hline
H & T & U &H & T & U, & T\up{th} & \textbf{H\up{th}} & K\up{th} \\
\hline
\hline
 &   & 5 & 0 & 0 & 0 & & &\\
 &   &   & 4 & 0 & 0 & & &\\
 & + &   &   & 1 & 0 & & &\\
 &   &   &   &   & 3 & & &\\
\hline
\hline
 &   & 5 & 4 & 1 & 3 & & &\\
\hline
\end{tabular}
\end{center}

\scriptsize
\begin{center}
% use packages: array
\begin{tabular}{ll}
H : Hundreds & T\up{th} : Tenths\\
T : Tens & H\up{th} :Hundredths\\
U : Units & K\up{e} : Thousandths\\
\end{tabular}
\end{center}

\normalsize
The answer is d).\\

7340-- Which number is represented by the addition?\\
\begin{center}
100 000 + 1000 + 100 + 10
\end{center}

a) 111\\
b) 11 110\\
c) 101 110\\
d) 111 111\\

Answer : c)\\

Feedback :\\
\begin{center}
% use packages: array
\begin{tabular}{|rrr|rrr|rrr|}
\hline
\multicolumn{6}{|c|}{integers} &\multicolumn{3}{|c|}{decimals} \\
\hline
\multicolumn{3}{|c|}{Class of} &\multicolumn{3}{|c|}{Class of} &  \multicolumn{3}{c|}{} \\
\multicolumn{3}{|c|}{thousands} &\multicolumn{3}{|c|}{units} &  \multicolumn{3}{c|}{} \\
\hline
H & T & U &H & T & U, & T\up{th} & \textbf{H\up{th}} & K\up{th} \\
\hline
\hline
1 & 0 & 0 & 0 & 0 & 0 & & & \\
  &   & 1 & 0 & 0 & 0 & & & \\
+ &   &   & 1 & 0 & 0 & & & \\
  &   &   &   & 1 & 0 & & & \\
\hline
\hline
1 & 0 & 1 & 1 & 1 & 0 & & & \\
\hline
\end{tabular}
\end{center}

\scriptsize
\begin{center}
% use packages: array
\begin{tabular}{ll}
H : Hundreds & T\up{th} : Tenths\\
T : Tens & H\up{th} :Hundredths\\
U : Units & K\up{e} : Thousandths\\
\end{tabular}
\end{center}

\normalsize
The answer is c).\\

7341-- Which number is represented by the addition?\\
\begin{center}
400 000 + 50 000 + 1000 + 800
\end{center}

a) 4518\\
b) 8154\\
c) 451 800\\
d) 800 154\\

Answer : c)\\

Feedback :\\
\begin{center}
% use packages: array
\begin{tabular}{|rrr|rrr|rrr|}
\hline
\multicolumn{6}{|c|}{integers} &\multicolumn{3}{|c|}{decimals} \\
\hline
\multicolumn{3}{|c|}{Class of} &\multicolumn{3}{|c|}{Class of} &  \multicolumn{3}{c|}{} \\
\multicolumn{3}{|c|}{thousands} &\multicolumn{3}{|c|}{units} &  \multicolumn{3}{c|}{} \\
\hline
H & T & U &H & T & U, & T\up{th} & \textbf{H\up{th}} & K\up{th} \\
\hline
\hline
4 & 0 & 0 & 0 & 0 & 0 & & &\\
  & 5 & 0 & 0 & 0 & 0 & & &\\
+ &   & 1 & 0 & 0 & 0 & & &\\
  &   &   & 8 & 0 & 0 & & &\\
\hline
\hline
4 & 5 & 1 & 8 & 0 & 0 & & &\\
\hline
\end{tabular}
\end{center}

\scriptsize
\begin{center}
% use packages: array
\begin{tabular}{ll}
H : Hundreds & T\up{th} : Tenths\\
T : Tens & H\up{th} :Hundredths\\
U : Units & K\up{e} : Thousandths\\
\end{tabular}
\end{center}

\normalsize
The answer is c).\\


7342-- Which number is represented by the addition?\\
\begin{center}
30 000 + 600 + 20
\end{center}

a) 263\\
b) 362\\
c) 30 026\\
d) 30 620\\

Answer : d)\\

Feedback :\\
\begin{center}
% use packages: array
\begin{tabular}{|rrr|rrr|rrr|}
\hline
\multicolumn{6}{|c|}{integers} &\multicolumn{3}{|c|}{decimals} \\
\hline
\multicolumn{3}{|c|}{Class of} &\multicolumn{3}{|c|}{Class of} &  \multicolumn{3}{c|}{} \\
\multicolumn{3}{|c|}{thousands} &\multicolumn{3}{|c|}{units} &  \multicolumn{3}{c|}{} \\
\hline
H & T & U &H & T & U, & T\up{th} & \textbf{H\up{th}} & K\up{th} \\
\hline
\hline
 & 3 & 0 & 0 & 0 & 0 & & & \\
 & + &   & 6 & 0 & 0 & & & \\
 &   &   &   & 2 & 0 & & & \\
\hline
\hline
 & 3 & 0 & 6 & 2 & 0 & & & \\
\hline
\end{tabular}
\end{center}

\scriptsize
\begin{center}
% use packages: array
\begin{tabular}{ll}
H : Hundreds & T\up{th} : Tenths\\
T : Tens & H\up{th} :Hundredths\\
U : Units & K\up{e} : Thousandths\\
\end{tabular}
\end{center}

\normalsize
The answer is d).\\


7343-- Which number is represented by the addition?\\
\begin{center}
5000 + 800 + 30 + 5
\end{center}

a) 5385\\
b) 5835\\
c) 8535\\
d) 583 500\\

Answer : b)\\

Feedback :\\
\begin{center}
% use packages: array
\begin{tabular}{|rrr|rrr|rrr|}
\hline
\multicolumn{6}{|c|}{integers} &\multicolumn{3}{|c|}{decimals} \\
\hline
\multicolumn{3}{|c|}{Class of} &\multicolumn{3}{|c|}{Class of} &  \multicolumn{3}{c|}{} \\
\multicolumn{3}{|c|}{thousands} &\multicolumn{3}{|c|}{units} &  \multicolumn{3}{c|}{} \\
\hline
H & T & U &H & T & U, & T\up{th} & \textbf{H\up{th}} & K\up{th} \\
\hline
\hline
 &   & 5 & 0 & 0 & 0 & & &\\
 &   &   & 8 & 0 & 0 & & &\\
 & + &   &   & 3 & 0 & & &\\
 &   &   &   &   & 5 & & &\\
\hline
\hline
 &   & 5 & 8 & 3 & 5 & & &\\
\hline
\end{tabular}
\end{center}

\scriptsize
\begin{center}
% use packages: array
\begin{tabular}{ll}
H : Hundreds & T\up{th} : Tenths\\
T : Tens & H\up{th} :Hundredths\\
U : Units & K\up{e} : Thousandths\\
\end{tabular}
\end{center}

\normalsize
The answer is b).\\

7344-- Which number is represented by th addition?\\
\begin{center}
300 000 + 80 000 + 5000 + 900 + 90 + 7
\end{center}

a) 9597\\
b) 88 097\\
c) 385 997\\
d) 799 583\\

Answer : c)\\

Feedback :\\
\begin{center}
% use packages: array
\begin{tabular}{|rrr|rrr|rrr|}
\hline
\multicolumn{6}{|c|}{integers} &\multicolumn{3}{|c|}{decimals} \\
\hline
\multicolumn{3}{|c|}{Class of} &\multicolumn{3}{|c|}{Class of} &  \multicolumn{3}{c|}{} \\
\multicolumn{3}{|c|}{thousands} &\multicolumn{3}{|c|}{units} &  \multicolumn{3}{c|}{} \\
\hline
H & T & U &H & T & U, & T\up{th} & \textbf{H\up{th}} & K\up{th} \\
\hline
\hline
3 & 0 & 0 & 0 & 0 & 0 & & &\\
  & 8 & 0 & 0 & 0 & 0 & & &\\
  &   & 5 & 0 & 0 & 0 & & &\\
  &   &   & 9 & 0 & 0 & & &\\
  &   &   &   & 9 & 0 & & &\\
  &   &   &   &   & 7 & & &\\
\hline
\hline
3 & 8 & 5 & 9 & 9 & 7 & & &\\
\hline
\end{tabular}
\end{center}

\scriptsize
\begin{center}
% use packages: array
\begin{tabular}{ll}
H : Hundreds & T\up{th} : Tenths\\
T : Tens & H\up{th} :Hundredths\\
U : Units & K\up{e} : Thousandths\\
\end{tabular}
\end{center}

\normalsize
The answer is c).\\

7345-- Which one is an even number?\\

a) 3985\\
b) 4914\\
c) 14 815\\
d) 420 423\\

Answer : b)\\

Feedback :\\
The number 4914 is evenly divisable by  2, because the units digit is divisable by 2 without remainder.\\
\begin{eqnarray*}
\frac{4}{2}=2\\
\end{eqnarray*}
Since 4914 is evenly divisable by 2, it is an even number.\\
The answer is b).\\


7346-- Which one is an odd number?\\

a) 4903\\
b) 4904\\
c) 4916\\
d) 4918\\

Answer : a)\\

Feedback :\\
The number 4903 is not evenly divisable by 2, because the units digit is not divisable by 2 without remainder.\\
\begin{eqnarray*}
\frac{3}{2}=1,5\\
\end{eqnarray*}
Since 4903 is not evenly divisable by 2, it is an odd number.\\
The answer is a).\\

7347-- What is the position of the digit 6 in the number 212,26?\\

a) hundreds\\
b) hundredths\\
c) tenths\\
d) units\\

Answer : b)\\

Feedback :\\
\begin{center}
% use packages: array
\begin{tabular}{|rrr|rrr|rrr|}
\hline
\multicolumn{6}{|c|}{integers} &\multicolumn{3}{|c|}{decimals} \\
\hline
\multicolumn{3}{|c|}{Class of} &\multicolumn{3}{|c|}{Class of} &  \multicolumn{3}{c|}{} \\
\multicolumn{3}{|c|}{thousands} &\multicolumn{3}{|c|}{units} &  \multicolumn{3}{c|}{} \\
\hline
H & T & U &H & T & U, & T\up{th} & \textbf{H\up{th}} & K\up{th} \\
\hline
\hline
& & & 2 & 1 & 2, & 2 & \textbf{6} &  \\
\hline
\end{tabular}
\end{center}

\scriptsize
\begin{center}
% use packages: array
\begin{tabular}{ll}
H : Hundreds & T\up{th} : Tenths\\
T : Tens & H\up{th} :Hundredths\\
U : Units & K\up{e} : Thousandths\\
\end{tabular}
\end{center}

\normalsize
The answer is b).\\


7348-- What is the position of digit 4 in the number 859,46?\\

a) hundreds\\
b) hundredths\\
c) tenths\\
d) tens\\

Answer : c)\\

Feedback :\\
\begin{center}
% use packages: array
\begin{tabular}{|rrr|rrr|rrr|}
\hline
\multicolumn{6}{|c|}{integers} &\multicolumn{3}{|c|}{decimals} \\
\hline
\multicolumn{3}{|c|}{Class of} &\multicolumn{3}{|c|}{Class of} &  \multicolumn{3}{c|}{} \\
\multicolumn{3}{|c|}{thousands} &\multicolumn{3}{|c|}{units} &  \multicolumn{3}{c|}{} \\
\hline
H & T & U &H & T & U, & T\up{th} & \textbf{H\up{th}} & K\up{th} \\
\hline
\hline
 & & & 8 & 5 & 9, & \textbf{4} & 6 & \\
\hline
\end{tabular}
\end{center}

\scriptsize
\begin{center}
% use packages: array
\begin{tabular}{ll}
H : Hundreds & T\up{th} : Tenths\\
T : Tens & H\up{th} :Hundredths\\
U : Units & K\up{e} : Thousandths\\
\end{tabular}
\end{center}

\normalsize
The answer is c).\\

7349-- What is the position of the digit 1 in the number 756,19?\\

a) Hundreds\\
b) hundredths\\
c) tenths\\
d) units\\

Answer : c)\\

Feedback :\\
\begin{center}
% use packages: array
\begin{tabular}{|rrr|rrr|rrr|}
\hline
\multicolumn{6}{|c|}{integers} &\multicolumn{3}{|c|}{decimals} \\
\hline
\multicolumn{3}{|c|}{Class of} &\multicolumn{3}{|c|}{Class of} &  \multicolumn{3}{c|}{} \\
\multicolumn{3}{|c|}{thousands} &\multicolumn{3}{|c|}{units} &  \multicolumn{3}{c|}{} \\
\hline
H & T & U &H & T & U, & T\up{th} & \textbf{H\up{th}} & K\up{th} \\
\hline
\hline
 & & & 7 & 5 & 6, & \textbf{1} & 9 & \\
\hline
\end{tabular}
\end{center}

\scriptsize
\begin{center}
% use packages: array
\begin{tabular}{ll}
H : Hundreds & T\up{th} : Tenths\\
T : Tens & H\up{th} :Hundredths\\
U : Units & K\up{e} : Thousandths\\
\end{tabular}
\end{center}

\normalsize
The answer is c).\\



\end{document} 
