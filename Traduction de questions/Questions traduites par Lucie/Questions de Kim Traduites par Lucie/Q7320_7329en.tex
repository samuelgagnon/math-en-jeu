\documentclass[letterpaper, 12pt]{article}
\usepackage[french]{babel}

\usepackage{amsmath,amsfonts,amsthm,amssymb,graphicx,multirow,hyperref,color}
\usepackage[latin1]{inputenc}

\pagestyle{plain}

\setlength{\topmargin}{-2cm}
\setlength{\textheight}{23.5cm}
\setlength{\textwidth}{12cm}
\setlength{\oddsidemargin}{-1cm}
\setlength{\parindent}{0pt}

\begin{document}

7320-- Jules has to travel 178 km by bike to get to his cabin. At noon he had travelled 63 km. How many kilometers 
does he still have to travel? Identify the operation you would need to solve this problem. \\

a) addition\\
b) division\\
c) multiplication\\
d) subtraction\\

Answer : d)\\

Feedback :\\
Jules has to travel a total of 178 km and he has already travelled 63. So he has \textbf{less} kilometers left to do than when he started.

\begin{center}
$178-63 = 115$
\end{center}
The answer is d).\\




7321-- The bicycle path in Boisjoli is 112 km long. There is a rest area every 14 kilometers. How many rest areas are there along this path? Identify the operation you would need to solve the problem. \\

a) addition\\
b) division\\
c) multiplication\\
d) subtraction\\

Answer : b)\\

Feedback :\\
By dividing the total number of kilometers by the number of kilometers that separate each rest area, we get the total number of rest areas.\\

\begin{eqnarray*}
\frac{112}{14}=8\\
\end{eqnarray*}
The answer is b).\\




7322-- Choose the most realistic number to complete the sentence.\\
\begin{center}
\'Emile got \underline{\quad\quad}\% on his writing exam.\\
\end{center}

a) -5\\
b) -0,25\\
c) 2$\frac{2}{3}$\\
d) 85\\

Answer : d)\\

Feedback :\\
\begin{center}
\'Emile got \textbf{85}\% on his writing exam.\\
\end{center}

Unless you have a very peculiar teacher, the results of an exam can't be a negative number nor a percentage over 100\%. In fact, a percentage (\%) is only a fraction of 100. We use it so often that mathematicians gave it a name.\\
The answer is d).\\

7323-- Choose the most realistic number to complete the sentence.\\ 
\begin{center}
Julie put \underline{\quad\quad} cups of brown sugar in her cookie recipe. \\
\end{center}

a) -1\\
b) -0,25\\
c) 2$\frac{1}{4}$\\
d) 500\\

Answer : c)\\

Feedback :\\
\begin{center}
Julie put \textbf{2$\frac{1}{4}$} cups of brown sugar in her cookie recipe.\\
\end{center}

First of all, negative numbers are eliminated. Next, the number 500 is way too big to be realistic.\\
The answer is c).\\



7324-- Choose the most realistic number to complete the sentence.\\ 
\begin{center}
Mary's little sister is celebrating her\underline{\quad\quad} $^{th}$ anniversary.\\
\end{center}

a) -1\\
b) -0,25\\
c) 2$\frac{1}{4}$\\
d) 11\\

Answer : d)\\

Feedback :\\
\begin{center}
Mary's little sister is celebrating her \textbf{11$^{th}$} anniversary.\\
\end{center}

First of all, negative numbers are not possible. Next, the fraction is to be eliminated because we don't celebrate
the fraction of a year.\\
The answer is d).\\


7325-- Choose the most realistic number to complete the sentence.\\
\begin{center}
A bubblegum costs \underline{\quad\quad}\$.\\
\end{center}

a) -5\\
b) 0,25\\
c) 20\\
d) 100\\

Answer : b)\\

Feedback :\\
\begin{center}
A bubblegum costs \textbf{0,25}\$.\\
\end{center}

First, prices are always positive numbers, so -5 is impossible. Also, the numbers 20 and 100 are way too high to be realistic.\\
The answer is b).\\


7326-- Place the number 983 502 in the correct set.

\begin{center}
    \includegraphics[width=6cm]{Q7326.eps}
% Q7326.eps : 300dpi, width=3.39cm, height=3.39cm, bb=0 0 400 400
    \end{center}


a) Set A\\
b) Set B\\
c) Set C\\
d) Set D\\

Answer : c)\\

Feedback :\\
\begin{center}
% use packages: array
\begin{tabular}{|rrr|rrr|rrr|}
\hline
\multicolumn{6}{|c|}{integers} &\multicolumn{3}{|c|}{decimals} \\
\hline
\multicolumn{3}{|c|}{Class of} &\multicolumn{3}{|c|}{Class of} &  \multicolumn{3}{c|}{} \\
\multicolumn{3}{|c|}{thousands} &\multicolumn{3}{|c|}{units} &  \multicolumn{3}{c|}{} \\
\hline
H & T & U &H & T & U, & T\up{th} & \textbf{H\up{th}} & K\up{th} \\
\hline
\hline
&  & 1 & 0 & 0 & 0 & 0 & 0 & 0 &  &  &  \\
 &  &  & 9 & 8 & 3 & 5 & 0 & 2  &  &  & \\
 &  &  &  &  &  &  &  & 0  &  &  & \\
\hline
\end{tabular}
\end{center}

\tiny
\begin{center}
% use packages: array
\begin{tabular}{ll}
H : Hundreds & T\up{th} : Tenths\\
T : Tens & H\up{th} :Hundredths\\
U : Units & K\up{e} : Thousandths\\
\end{tabular}
\end{center}

\normalsize

To begin with, since the number 983 502 is smaller than \mbox{1 000 000}, it is part of set A.\\

Next, the number 983 502 does not have a 5 in the hundreds of thousands position and does not have a 2 in the units position.\\

The number 983 502 is then part of set C.\\
The answer is c).\\


7327-- Which number has the highest amount of tens of thousands? \\

a) 193 000\\
b) 560 000\\
c) 851 888\\
d) 910 124\\

Answer : d)\\

Feedback :\\

\begin{center}
% use packages: array
\begin{tabular}{|rrr|rrr|rrr|}
\hline
\multicolumn{6}{|c|}{integers} &\multicolumn{3}{|c|}{decimals} \\
\hline
\multicolumn{3}{|c|}{Class of} &\multicolumn{3}{|c|}{Class of} &  \multicolumn{3}{c|}{} \\
\multicolumn{3}{|c|}{thousands} &\multicolumn{3}{|c|}{units} &  \multicolumn{3}{c|}{} \\
\hline
H & T & U &H & T & U, & T\up{th} & \textbf{H\up{th}} & K\up{th} \\
\hline
\hline
 \textbf{1} & \textbf{9} & 3 & 0 & 0 & 0 & & & \\
 \textbf{5} & \textbf{6} & 0 & 0 & 0 & 0 & & &\\
 \textbf{8} & \textbf{5} & 1 & 8 & 8 & 8 & & &\\
 \textbf{9} & \textbf{1} & 0 & 1 & 2 & 4 & & &\\
\hline
\end{tabular}
\end{center}

\tiny
\begin{center}
% use packages: array
\begin{tabular}{ll}
H : Hundreds & T\up{th} : Tenths\\
T : Tens & H\up{th} :Hundredths\\
U : Units & K\up{e} : Thousandths\\
\end{tabular}
\end{center}

\normalsize

The number 910 124 has 91 tens of thousands.\\
The answer is d).\\



7328--Which number represents \textit{nine hundred twenty nine thousand and three hundred}? \\

a) 129 300\\
b) 222 222\\
c) 300 929\\
d) 929 300\\

Answer : d)\\

Feedback :\\
\begin{center}
% use packages: array
\begin{tabular}{|rrr|rrr|rrr|}
\hline
\multicolumn{6}{|c|}{integers} &\multicolumn{3}{|c|}{decimals} \\
\hline
\multicolumn{3}{|c|}{Class of} &\multicolumn{3}{|c|}{Class of} &  \multicolumn{3}{c|}{} \\
\multicolumn{3}{|c|}{thousands} &\multicolumn{3}{|c|}{units} &  \multicolumn{3}{c|}{} \\
\hline
H & T & U &H & T & U, & T\up{th} & \textbf{H\up{th}} & K\up{th} \\
\hline
\hline
\multicolumn{3}{|c|}{neuf cent} &  \multicolumn{3}{|c|}{} & & &\\
\multicolumn{3}{|c|}{vingt-neuf} &  \multicolumn{3}{|c|}{trois cents} & & &\\
\multicolumn{3}{|c|}{mille} &  \multicolumn{3}{|c|}{} & & &\\
 9 & 2 & 9 & 3 & 0 & 0 & & &\\
\hline
\end{tabular}
\end{center}

\tiny
\begin{center}
% use packages: array
\begin{tabular}{ll}
H : Hundreds & T\up{th} : Tenths\\
T : Tens & H\up{th} :Hundredths\\
U : Units & K\up{e} : Thousandths\\
\end{tabular}
\end{center}

\normalsize

The answer is d).\\




7329--Round each number to the nearest unit and do the operation.\\

\begin{eqnarray*}
9,64 \times 2,01\\
\end{eqnarray*}

a) 2\\
b) 9\\
c) 12\\
d) 20\\

Answer: d)

Feedback :\\
First, we round up the number 9,64 to 10, because the number in the tenths position is greater or equal to 5.\\

Next, we round down the number 2,01 to 2, because the number in the tenths position is smaller than 5.\\

Finally, we do the multiplication.\\

\begin{center}
 \begin{tabular}{r r}
 & 10\\
$\times$ & 2\\
\hline
 & 20
\end{tabular}
\end{center}
The answer is d).\\





\end{document}










