\documentclass[letterpaper, 12pt]{article}
\usepackage[french]{babel}

\usepackage{amsmath,amsfonts,amsthm,amssymb,graphicx,multirow,hyperref,color}
\usepackage[latin1]{inputenc}

\pagestyle{plain}

\setlength{\topmargin}{-2cm}
\setlength{\textheight}{23.5cm}
\setlength{\textwidth}{12cm}
\setlength{\oddsidemargin}{-1cm}
\setlength{\parindent}{0pt}

\begin{document}

7370-- Complete the sentence. \\
\begin{center}
Any number is divisible by five without remainder if the last digit of the number is \underline{\quad\quad}.\\
\end{center}

a) five or seven\\
b) one or five\\
c) one or two\\
d) zero or five\\

Answer : d)\\

Feedback :
\begin{center}
Any number is divisible by five without remainder if the last digit of the number is \textbf{zero or five}.\\
\end{center}

Example 1 : 82 235\\
\begin{eqnarray*}
\frac{35}{5}=7
\end{eqnarray*}
\begin{eqnarray*}
\frac{82\ 235}{5}= 16\ 447
\end{eqnarray*}


Example 2 : 968 480\\
\begin{eqnarray*}
\frac{80}{5}=16
\end{eqnarray*}
\begin{eqnarray*}
\frac{968\ 480}{5}= 193\ 696
\end{eqnarray*}


The answer is d).\\

7371-- Complete the sentence.\\
\begin{center}
Any number is divisible by six without remainder, if the last digit of the number is evenly divisible by \underline{\quad\quad} and if the sum of its digits is evenly divisible by \underline{\quad\quad}.\\
\end{center}

a) two, three\\
b) seven, nine\\
c) six, five\\
d) three, eight\\

Answer : a)\\

Feedback :
\begin{center}
Any number is divisible by six without remainder if the last digit is evenly divisible byTout nombre est divisible by \textbf{two} and if the sum of its digits is evenly divisible by \textbf{three}.\\
\end{center}

Example: 16 212\\
\begin{eqnarray*}
\frac{2}{2}=1
\end{eqnarray*}
\begin{eqnarray*}
1+6+2+1+2 = 12
\end{eqnarray*}
\begin{eqnarray*}
\frac{12}{3}=4
\end{eqnarray*}
\begin{eqnarray*}
\frac{16\ 212}{6}= 2702
\end{eqnarray*}
The answer is a).\\

7372-- Complete the sentence.\\
\begin{center}
Any number is divisible by eight without remainder if the number formed by the \underline{\quad\quad} last digits is evenly divisible by eight.\\
\end{center}

a) two \\
b) five\\
c) four\\
d) three\\

Answer : d)\\

Feedback :
\begin{center}
Any number is divisible by eight without remainder if the number formed by the \textbf{three} last digits is evenly divisible by eight.\\
\end{center}

Example : 655 456\\
\begin{eqnarray*}
\frac{456}{8}=57
\end{eqnarray*}
\begin{eqnarray*}
\frac{655\ 456}{8}=81\ 932
\end{eqnarray*}
The answer is d).\\

7373-- Complete the sentence.\\
\begin{center}
Any number is divisible by nine without remainder if the sum of its digits is evenly divisible by \underline{\quad\quad}.\\
\end{center}

a) five \\
b) nine\\
c) seven\\
d) one\\

Answer : b)\\

Feedback :
\begin{center}
Any number is divisible by nine without remainder if the sum of its digits is evenly divisible by \textbf{nine}.\\
\end{center}

Example : 32 526\\
\begin{eqnarray*}
3+2+5+2+6=18
\end{eqnarray*}
\begin{eqnarray*}
\frac{18}{9}=2
\end{eqnarray*}
\begin{eqnarray*}
\frac{32\ 526}{9}=3614
\end{eqnarray*}

The answer is b).\\

7374-- Complete the sentence.\\
\begin{center}
Any number is divisible by ten without remainder if the last digit is \underline{\quad\quad}.\\
\end{center}

a) five\\
b) two\\
c) four\\
d) zero\\

Answer : d)\\

Feedback :
\begin{center}
Any number is divisible by ten without remainder if its last digit is \textbf{zero}.\\
\end{center}

Example : 896 260\\

\begin{eqnarray*}
\frac{896\ 260}{10}=89\ 626
\end{eqnarray*}

The answer is d).\\


7375-- Which number is divisible by 2 without remainder? \\

a) 6441\\
b) 76 331\\
c) 873 458\\
d) 987 455\\

Answer: c)\\

Feedback :\\
The number 873 458 is divisible by 2 without remainder because the units digit is evenly divisible by 2. \\

\begin{eqnarray*}
\frac{8}{2}=4
\end{eqnarray*}
\begin{eqnarray*}
\frac{873\ 458}{2}=436\ 729
\end{eqnarray*}

The answer is c).\\


7376-- Which number is divisible by 2 without remainder? \\

a) 100 016\\
b) 100 315\\
c) 119 421\\
d) 123 335\\

Answer : a)\\

Feedback :\\
The number 100 016 is divisible by 2 without remainder because the units digit is evenly divisible by 2. \\

\begin{eqnarray*}
\frac{6}{2}=3
\end{eqnarray*}
\begin{eqnarray*}
\frac{100\ 016}{2}=50\ 008
\end{eqnarray*}

The answer is a).\\


7377-- Which number is divisible by 5 without remainder? \\

a) 100 016\\
b) 100 315\\
c) 119 421\\
d) 123 337\\

Answer : b)\\

Feedback :\\
The number 100 315 is divisible by 5 without remainder because the units digit is 5. \\

\begin{eqnarray*}
\frac{100\ 315}{5}=20\ 063
\end{eqnarray*}

The answer is b).\\

7378-- Which number is divisible by 5 without remainder? \\

a) 5810\\
b) 6316\\
c) 8421\\
d) 8637\\

Answer : a)\\

Feedback :\\
The number 5810 is divisible by 5 without remainder because the units digit is 0. \\

\begin{eqnarray*}
\frac{5810}{5}=1162
\end{eqnarray*}

The answer is a).\\


7379--  Simplify the fraction.\\
\begin{eqnarray*}
\frac{3}{9}\\
\end{eqnarray*}

a) $\frac{1}{9}$\\

b) $\frac{2}{9}$\\

c) $\frac{1}{3}$\\

d) $\frac{2}{3}$\\

Answer : c)\\

Feedback :\\
\begin{center}
\begin{tabular}{c}

$\div 3$  \\
\LARGE $\curvearrowright$  \\
\Large $\frac{3}{9}$  =  \Large $\frac{1}{3}$ \\
\rotatebox{180}{\LARGE$\curvearrowleft$}\\
$\div 3$  \\

\end{tabular}
\end{center}

The answer is c).\\

7380--  Perform the addition.\\
\begin{eqnarray*}
\frac{1}{8}+\frac{4}{8}\\
\end{eqnarray*}

a) $\frac{3}{16}$\\

b) $\frac{5}{16}$\\

c) $\frac{3}{8}$\\

d) $\frac{5}{8}$\\

Answer : d)\\

Feedback :
\begin{eqnarray*}
\frac{1}{8}+\frac{4}{8}=\frac{5}{8}
\end{eqnarray*}

The answer is d).\\


7381--  The Eternal Snow ski resort, sold 100 season tickets at 200\$ each. How much money did the resort make with these tickets?\\

a) 100\$\\
b) 200\$\\
c) 2000\$\\
d) 20 000\$\\

Answer : d)\\

Feedback :
\begin{center}
% use packages: array
\begin{tabular}{rrrrrr}
& & & 1 & 0 & 0\\
+& x &  & 2 & 0 & 0\\
\cline{2-6}
& & & 0 & 0 & 0\\
& & 0 & 0 & 0 & \\
& 2 & 0 & 0 &  & \\
\hline
\hline
& 2 & 0 & 0 & 0 & 0\\

\end{tabular}
The answer is d).\\


7382-- Choose the pie chart that corresponds to the statement.\\
\begin{center}
At Christopher's school, 25\% of the teachers are men.\\
\end{center}

a)\\
\includegraphics[width=2cm]{Q7382a.eps}
% Q7382a.eps : 300dpi, width=3.39cm, height=3.39cm, bb=0 0 400 400
\\

b)\\
\includegraphics[width=2cm]{Q7382b.eps}
% Q7382b.eps : 300dpi, width=3.39cm, height=3.39cm, bb=0 0 400 400
\\

c)\\
\includegraphics[width=2cm]{Q7382c.eps}
% Q7382c.eps : 300dpi, width=3.39cm, height=3.39cm, bb=0 0 400 400
\\

d)\\
\includegraphics[width=2cm]{Q7382d.eps}
% Q7382d.eps : 300dpi, width=3.39cm, height=3.39cm, bb=0 0 400 400
\\

Answer : a)\\

Feedback :\\
\begin{eqnarray*}
25\% = \frac{25}{100}\\
\end{eqnarray*}

\begin{center}
\begin{tabular}{c}
$\div 25$  \\
\LARGE $\curvearrowright$  \\
\Large $\frac{25}{100}$  =  \Large $\frac{1}{4}$ \\
\rotatebox{180}{\LARGE$\curvearrowleft$}\\
$\div 25$  \\
\end{tabular}
\end{center}

\begin{center}
\includegraphics[width=2cm]{R7382.eps}
% R7382.eps : 300dpi, width=3.39cm, height=3.39cm, bb=0 0 400 400
\end{center}

The answer is a).\\

7383--  Julien gives cubes of 1 cm$^{3}$ to four of his students.\\

\begin{center}
% use packages: array
\begin{tabular}{|c|c|}
\hline
 Students & Number of cubes\\
\hline
\hline
Cendrine & 8\\
\hline
Nino & 9\\
\hline
Rose & 5\\
\hline
Sylvaine & 2\\
\hline
\end{tabular}
\end{center}

Which of these students can form a square pyramid with the cubes?\\

a) Cendrine\\
b) Nino\\
c) Rose\\
d) Sylvaine\\

Answer : b)\\

Feedback :\\
\begin{center}
\includegraphics[width=5cm]{Q7383.eps}
% Q7383.eps : 300dpi, width=3.39cm, height=3.39cm, bb=0 0 400 400
\end{center}
\begin{eqnarray*}
3 \times 3=9\\
3^{2}=9\\
\end{eqnarray*}
The answer is b).\\


7384--  Julien distributes cubes of 1 cm$^{3}$ to four of his students.\\

\begin{center}
% use packages: array
\begin{tabular}{|c|c|}
\hline
 Students & Number of cubes\\
\hline
\hline
Cendrine & 8\\
\hline
Nino & 9\\
\hline
Rose & 5\\
\hline
Sylvaine & 2\\
\hline
\end{tabular}
\end{center}

Which of the students can create a cube with all of his or her small cubes?\\

a) Cendrine\\
b) Nino\\
c) Rose\\
d) Sylvaine\\

Answer : a)\\

Feedback :\\
\begin{center}
\includegraphics[width=5cm]{Q7384.eps}
% Q7384.eps : 300dpi, width=3.39cm, height=3.39cm, bb=0 0 400 400
\end{center}
\begin{eqnarray*}
2\times 2\times 2=8\\
2^{3}=8\\
\end{eqnarray*}
The answer is a).\\

7385--  What space do Jolaine's socks occupy in her dresser?\\

\begin{center}
\includegraphics[width=4cm]{Q7385.eps}
% Q7385.eps : 300dpi, width=3.39cm, height=3.39cm, bb=0 0 400 400
\end{center}

a) $\frac{1}{6}$\\

b) $\frac{1}{3}$\\

c) $\frac{2}{6}$\\

d) $\frac{2}{3}$\\

Answer : a)\\

Feedback :\\
\begin{center}
\includegraphics[width=5cm]{R7385.eps}
% R7385.eps : 300dpi, width=3.39cm, height=3.39cm, bb=0 0 400 400
\end{center}
The answer is a).\\

7386--  What space do Jolaine's pyjamas occupy in her dresser?\\

\begin{center}
\includegraphics[width=4cm]{Q7386.eps}
% Q7386.eps : 300dpi, width=3.39cm, height=3.39cm, bb=0 0 400 400
\end{center}

a) $\frac{1}{6}$\\

b) $\frac{1}{3}$\\

c) $\frac{4}{6}$\\

d) $\frac{2}{3}$\\

Answer : b)\\

Feedback :\\
\begin{center}
\includegraphics[width=5cm]{R7386.eps}
% R7386.eps : 300dpi, width=3.39cm, height=3.39cm, bb=0 0 400 400
\end{center}
The answer is b).\\


7387--  What space do Jolaine's socks and sweaters occupy in her dresser?\\

\begin{center}
\includegraphics[width=4cm]{Q7387.eps}
% Q7387.eps : 300dpi, width=3.39cm, height=3.39cm, bb=0 0 400 400
\end{center}

a) $\frac{1}{6}$\\

b) $\frac{1}{3}$\\

c) $\frac{3}{6}$\\

d) $\frac{2}{3}$\\

Answer : b)\\

Feedback :\\
\begin{center}
\includegraphics[width=5cm]{R7387.eps}
% R7387.eps : 300dpi, width=3.39cm, height=3.39cm, bb=0 0 400 400
\end{center}
The answer is b).\\

7388--  What space do Jolaine's pants and pyjamas occupy in her dresser?\\

\begin{center}
\includegraphics[width=4cm]{Q7388.eps}
% Q7388.eps : 300dpi, width=3.39cm, height=3.39cm, bb=0 0 400 400
\end{center}

a) $\frac{1}{6}$\\

b) $\frac{1}{3}$\\

c) $\frac{2}{6}$\\

d) $\frac{2}{3}$\\

Answer : d)\\

Feedback :\\
\begin{center}
\includegraphics[width=4cm]{R7388.eps}
% R7388.eps : 300dpi, width=3.39cm, height=3.39cm, bb=0 0 400 400
\end{center}
The answer is d).\\

7389--  What space do Jolaine's sweaters and pants occupy in her dresser?\\

\begin{center}
\includegraphics[width=4cm]{Q7389.eps}
% Q7389.eps : 300dpi, width=3.39cm, height=3.39cm, bb=0 0 400 400
\end{center}

a) $\frac{1}{6}$\\

b) $\frac{1}{3}$\\

c) $\frac{1}{2}$\\

d) $\frac{2}{3}$\\

Answer : c)\\

Feedback :\\
\begin{center}
\includegraphics[width=5cm]{R7389.eps}
% R7389.eps : 300dpi, width=3.39cm, height=3.39cm, bb=0 0 400 400
\end{center}
The answer is c).\\

7390--  What space do Jolaine's socks, sweaters and pants occupy in her dresser?\\

\begin{center}
\includegraphics[width=4cm]{Q7390.eps}
% Q7390.eps : 300dpi, width=3.39cm, height=3.39cm, bb=0 0 400 400
\end{center}

a) $\frac{1}{6}$\\

b) $\frac{1}{3}$\\

c) $\frac{1}{2}$\\

d) $\frac{2}{3}$\\

Answer : d)\\

Feedback :\\
\begin{center}
\includegraphics[width=5cm]{R7390.eps}
% R7390.eps : 300dpi, width=3.39cm, height=3.39cm, bb=0 0 400 400
\end{center}
The answer is d).\\

7391--  Complete the factor tree.\\
\begin{center}
\includegraphics[width=5cm]{Q7391.eps}
% Q7391.eps : 300dpi, width=3.39cm, height=3.39cm, bb=0 0 400 400
\end{center}

a) 2 and 18\\
b) 3 and 12\\
c) 4 and 9\\
d) 6 and 6\\

Answer : d)\\

R�troaction :\\
\begin{center}
\includegraphics[width=5cm]{R7391.eps}
% R7391.eps : 300dpi, width=3.39cm, height=3.39cm, bb=0 0 400 400
\end{center}
The answer is d).\\

7392--  A survey on student's favorite sports was conducted in Ma�ka's classroom. All the students gave an answer. Here are the results:\\

\begin{center}
% use packages: array
\begin{tabular}{|c|c|}
\hline
hockey & 8 students\\
\hline
figure skating & 6 students\\
\hline
ping pong & 3 students\\
\hline
martial arts & 2 students\\
\hline
soccer & 1 student\\
\hline
\end{tabular}
\end{center}

What is the percentage of the students who prefer hockey?\\

a) 5\% \\
b) 30\% \\
c) 40\% \\
d) 100\%, because everyone loves hockey!\\

Answer : c)\\
First, there are 8 students on a total of 20 that prefer hockey. To find the percentage, we can find the fraction (on a hundred) that is equivalent to $\frac{8}{20}$.\\
\begin{center}
\begin{tabular}{c}

$\times 5$  \\
\LARGE $\curvearrowright$  \\
\Large $\frac{8}{20}$  =  \Large $\frac{40}{100}$ \\
\rotatebox{180}{\LARGE$\curvearrowleft$}\\
$\times 5$  \\

\end{tabular}
\end{center}

Then, we transform the fraction into a pourcentage.\\

\begin{eqnarray*}
\frac{40}{100} = 40\% \\
\end{eqnarray*}
Feedback :\\

The answer is c).\\

7393-- Put the numbers in ascending order.\\
\begin{center}
4,5\ \ \ 4,10\ \ \ 4,12\\
\end{center}

a) 4,10\ \ \ 4,12\ \ \ 4,5\\
b) 4,12\ \ \ 4,5\ \ \ 4,10\\
c) 4,5\ \ \ 4,10\ \ \ 4,12\\
d) 4,5\ \ \ 4,12\ \ \ 4,10\\

Answer : a)\\

Feedback :\\
\begin{center}
% use packages: array
\begin{tabular}{|rrr|rrr|rrr|rrr|}
\hline
\multicolumn{6}{|c|}{integers} &\multicolumn{3}{|c|}{decimals} \\
\hline
\multicolumn{3}{|c|}{Class of} &\multicolumn{3}{|c|}{Class of} &  \multicolumn{3}{c|}{} \\
\multicolumn{3}{|c|}{thousands} &\multicolumn{3}{|c|}{units} &  \multicolumn{3}{c|}{} \\
\hline
H & T & U &H & T & U, & T\up{th} & \textbf{H\up{th}} & K\up{th} \\
\hline
\hline
& & & & & &  &  & 4, & 1 & 0 &\\
& & & & & &  &  & 4, & 1 & 2 &\\
& & & & & &  &  & 4, & 5 &   & \\
\hline
\end{tabular}
\end{center}

\scriptsize
\begin{center}
% use packages: array
\begin{tabular}{ll}
H : Hundreds & T\up{th} : Tenths\\
T : Tens & H\up{th} :Hundredths\\
U : Units & K\up{e} : Thousandths\\
\end{tabular}
\end{center}

\normalsize
The answer is a).\\


7394-- Jeremy paid 60\$ for video games that were sold for 20\$ each. How many games did he buy?\\

a) 3\\
b) 4\\
c) 30\\
d) 80\\

Answer : a)\\

Feedback :\\
By divising the total amount spent by the cost for one game, we get the number of games.\\

\begin{eqnarray*}
60�20=3\\
\end{eqnarray*}
The answer is a).\\

7395-- Charli is ten years old. If Mia is six years older than Charli, how old is she?\\

a) 4\\
b) 6\\
c) 10\\
d) 16\\

Answer : d)\\

Feedback :\\
Mia is six years \textbf{oler} than Charli.
\begin{center}
$10 + 6 = 16$
\end{center}
The answer is d).\\

7396-- How much will Harry have to pay for 3 sweaters at 25\$ each?\\

a) 25\$\\
b) 50\$\\
c) 75\$\\
d) 100\$\\

Answer : c)\\

Feedback :\\
For 3 sweaters at 25\$ each, Harry will have to pay 3 \textbf{times} the amount of 25\$, so 75\$.\\
\begin{center}
$3\times25 = 75$
\end{center}
The answer is c).\\

7397-- For Christmas, Maya got six gifts. If Isa\"i got four more than Maya, How many gifts did he get?\\

a) 2\\
b) 10\\
c) 18\\
d) 24\\

Answer : b)\\

Feedback :\\
Isa\"i got four gifts \textbf{more} than Maya.
\begin{center}
$6+4 = 10$
\end{center}
The answer is b).\\

7398-- Julius has to ride for 178 km by bike to get to his camp. At noon, he had done 63 km. How many more does he still have to do?\\

a) 115\\
b) 155\\
c) 221\\
d) 241\\

Answer : a)\\

Feedback :\\
Julius has to ride for 178 km and he has already done 63. So, he has \textbf{less} kilometres left to complete than at his starting point.

\begin{center}
$178-63 = 115$
\end{center}
The answer is a).\\


7399-- The bicycle path in Boisjoli is 112 km long. There is a rest area every 14 km. How many rest areas are there on the path?\\

a) 8\\
b) 98\\
c) 126\\
d) 1568\\

Answer : a)\\

Feedback :\\
By divising the total amount of kilometres by the number of kilometres between each rest areas, we get the total amount of rest areas.\\

\begin{eqnarray*}
\frac{112}{14}=8\\
\end{eqnarray*}
The answer is a).\\



\end{document} 